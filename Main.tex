\documentclass[a4paper, 12pt]{report}
%    \renewcommand{\baselinestretch}{1.6}      % interline spacing
%
% \includeonly{}
%
%			PREAMBOLO
%
\usepackage[pdfa]{hyperref}
\usepackage[a-2u]{pdfx}  

\usepackage[a4paper]{geometry}
\usepackage{amssymb,amsmath,amsthm}
\usepackage{graphicx}
\usepackage{url}
\usepackage{hyperref}
\usepackage{epsfig}
\usepackage[english]{babel}
\usepackage{setspace}
\usepackage{tesi}
\usepackage{listings}
\usepackage{xcolor,colortbl}

% per definitions
\usepackage{amsthm}
\theoremstyle{definition}
\newtheorem{defn}{Definition} % definition numbers are dependent on theorem numbers
\newtheorem{example}{Example} % definition numbers are dependent on theorem numbers



\usepackage{pifont}% http://ctan.org/pkg/pifont
\newcommand{\cmark}{\ding{51}}%
\newcommand{\xmark}{\ding{55}}%


% per le accentate
\usepackage[utf8]{inputenc}
%
\newtheorem{myteor}{Teorema}[section]
%
\newenvironment{teor}{\begin{myteor}\sl}{\end{myteor}}
%
%Path relative to the main .tex file 
\graphicspath{ {./images/} }
%
%			TITOLO
%
\begin{document}
\title{Design and development of an assurance methodology for security certifications in highly dynamic architectures}
\author{Matteo CAVAGNINO}
\dept{Corso di Laurea in Informatica} 
\anno{2021-2022}
\matricola{961707}
\relatore{Prof. Claudio A. ARDAGNA}
\correlatore{Dr. Nicola BENA}
%
%        \submitdate{month year in which submitted to GPO}
%		- date LaTeX'd if omitted
%	\copyrightyear{year degree conferred (next year if submitted in Dec.)}
%		- year LaTeX'd (or next year, in December) if omitted
%	\copyrighttrue or \copyrightfalse
%		- produce or don't produce a copyright page (false by default)
%	\figurespagetrue or \figurespagefalse
%		- produce or don't produce a List of Figures page
%		  (false by default)
%	\tablespagetrue or \tablespagefalse
%		- produce or don't produce a List of Tables page
%		  (false by default)
% 
\beforepreface % Frontespizio
%
%			DEDICA
% \prefacesection{}
%         {\hfill \Large {\sl dedicato a \dots}}
% 
%			PREFAZIONE
%
\prefacesection{Preface}
In recent years, the continuous research and development of increasingly dynamic and pervasive architectures led to new, more distributed and efficient services. However, this development has been met with a decreasing general (perceived) trustworthiness due to the direction needed to provide a better quality of service. The evolution from monolithic and static web services to the micro-services provided by the cloud allowed the understanding of how dynamic infrastructures drastically increase the general flexibility of a system; more specifically, service providers need not worry about the infrastructure but only about the software and end-users can benefit from a better and faster service granted by powerful server machines. Following this line of development, the Edge computing paradigm improved the concept of decentralization, allowing another technology to develop in functionalities, the Internet of Things (IoT). The development path led to networks and systems composed of high numbers of low-power devices, which make it difficult to test and certify their security features. IoT systems heavily rely on the possibility of adding, removing and relocating the numerous devices in their network, which can reach the thousands, without ever stopping the entire system; such configuration changes often imply relative software updates. On the other hand, ICT systems prove their assurance by means of certification, which implies long and heavy processes to release a certificate, which can easily be invalidated by any change in the configuration of the certified product.

The goal of this thesis is to provide a first approach to this problem, allowing highly dynamic systems to perform small changes in their configuration without invalidating the whole certificate, and quickly certifying only the needed properties with the minimum necessary effort.
%
%
%			ORGANIZZAZIONE
\section*{Organization of the thesis}
\label{organizzazione}
The thesis is organized as follows:
\begin{description}
    \item[Chapter 1] presents an introduction to the context of the thesis, describing a brief overview of today's dynamic technologies and the new assurance techniques, and introducing the approach taken in the following chapters of the thesis work.

    \item[Chapter 2] illustrates an in-depth overview of the modern architecture paradigms, starting from the Cloud and moving to more dynamic and decentralized systems like Edge and Internet of Things; furthermore, it follows a detailed summary of the modern standards and research in the assurance and certification field of such architectures, leaving space for a curated description of the problem, our work aims to solve.

    \item[Chapter 3] explains the proposed methodology to develop the proposed solution, introducing the formal descriptions and models of the certification elements that compose the final scheme. Compared to the traditional certification schemes, our proposal introduces two new components: the scoring system, a feature that allows guiding the certification process automatically, and the trigger, a component in charge of monitoring the changes carried to the system and signal manual operators in case such changes should affect any security property. Furthermore, this chapter presents adjusted models for the remaining components, such as properties, attributes, tests and certificates, allowing for better integration in the certification scheme. Finally, this chapter shows how such components should be operated and the order of procedures intended for the expected results, leading to the release of a partial certificate complementing the already possessed certificates and substituting the invalidated portions.

    \item[Chapter 4] exemplifies a complete procedure with the use of the proposed certification scheme. The process is executed over a real-world-inspired scenario, consisting of a web service implementing open-source OpenSSL. In this chapter, we show how such a target could be firstly certified using a traditional certification scheme (e.g. Common Criteria) and then, after two security patches, how the certificate could be maintained using our proposed scheme, drastically reducing the redundancy of a new certification process from scratch.

    \item[Chapter 5] finally concludes this thesis' work, illustrating the results achieved and giving inspiration on how to improve and develop the proposed certification scheme further.
    
    \item[Appendix A] lists all non-functional security requirements that IoT system manufacturers should consider when developing their infrastructures; such requirements are officially recommended by the National Institute of Standards and Technology (NIST) \cite{nist_req}.
    
    \item[Appendix B] presents the code used in Chapter 4 for testing the proposed certification scheme's evaluation target.
\end{description}
%
%			RINGRAZIAMENTI
%
\prefacesection{Acknowledgements}

I would like to thank my supervisor Prof. Claudio Ardagna and my co-supervisor, Dr Nicola Bena, who provided me with encouragement and patience throughout the duration of this project.

I also wish to thank my family and friends for supporting me throughout my entire academic journey.

Last but not least, I’m extremely grateful to my girlfriend Benedetta, that never wavered in her support and helped me achieve these results.
\begin{flushright}
\textit{Matteo Cavagnino}
\end{flushright}
\afterpreface
% 
% 
%			CAPITOLO 1: Introduzione
\chapter{Introduction}
\label{cap1}                        % riferire al capitolo con \ref{cap1}
The first companies that took advantage of the Internet to provide web services built their business around small, static server machines with limited computing power that offered little to no flexibility and required the manufacturer to access the remote machine and update it manually when needed. Hence, after years of improvements and solutions in favour of higher flexibility, the research and development direction aligned toward new technologies, such as the Cloud.
The Cloud computing paradigm represented a true innovation for web services providers since it granted virtually infinite resources, an excellent and maintained complete infrastructure, the remote management of the rented machines and different payment options that allowed customers to have more or fewer responsibilities over their systems. In addition, the dynamicity provided by the Cloud allowed for more flexible designs for web services and started a research path that inevitably led to more dynamic, flexible and pervasive architectures. The first major step in such a direction was toward a computational model closer to the end-users. Many online services started having strict latency requirements that the Cloud could not satisfy due to the physical distance between the server and the users. Furthermore, a new technology started diffusing, the Internet of Things, which involved a huge amount of devices with integrated sensors capable of producing immense moles of data that soon showed the Cloud's resources are finite and limited.

The first solid solution to the abovementioned problems is the Edge computing paradigm, which introduced the concept of computing nodes, small interconnected servers in charge of aggregating and pre-processing the data generated by the adjacent smart devices. Of course, such architecture still relies on the Cloud for the final data processing since the nodes are not powerful enough to do that in real-time, but it considerably simplifies the work for the Cloud.

During all the technological development over the years, one point always stood up as fundamental for the online services providers, the security of such technologies, from the software to the hardware infrastructures. Enterprise web services have faced continuous security threats since they began spreading across the Internet; additionally, web services started integrating other web services unifying the functionalities and features and expanding increasingly. The more integrated services, the heavier the work to ensure the integration was secure for the customers' devices and internal components such as databases and source code; moreover, laws started to become more strict over cyber-security and data protection progressively and started demanding proof of services' security. For these reasons, cyber-security certification schemes were introduced, allowing web services to obtain a certificate they could show proving the security of the system. Besides, such schemes underwent numerous iterations, eventually leading to a relatively low number of internationally accepted standards, the most known being the Common Criteria \cite{infrastructure2002common}. The certification schemes focused on demonstrating the correct behaviour of the various security components of the systems, including, but not limited to, connection protocols, data management systems, encryption algorithms and user privacy policies.
Subsequently, the focus moved from static web services to more dynamic infrastructures, such as the Cloud \cite{mell2011nist}, that needed more automated schemes for the certification processes; many researchers worked on developing new solutions for better certification schemes for the Cloud specifically, such as Anisetti et al. in \cite{anisetti2017semi}, facing the challenges brought up by the new environments, such as i) the contextual changes that might invalidate the traditional certificates, ii) the numerous layers of services (e.g. networking infrastructure, servers, virtual environments, development tools and included software) and iii) the incremental nature of Cloud systems, where new services are continuously added. Dynamicity means flexibility for the end user, and starting from the Cloud, the focus was moved to increasingly dynamic architectures, such as Edge computing and the Internet of Things. Edge (or Fog) computing added a number of challenges to the security assurance of systems; more specifically, the high number of devices with low resources involved was not an easy task to deal with, and it is not a fully mature topic yet. Researchers like Aslam et al. in \cite{aslam2020fonac} addressed some of such issues by introducing less invasive techniques like continuous monitoring and auditing of the Edge nodes. Following the trend of increasing the dynamicity and flexibility of systems, the Internet of Things (IoT) has been ramping up in popularity in recent years, dealing with increasingly sensitive data. IoT systems can reach tens of thousands of interconnected devices, and when such devices handle data that must stay private and protected, they easily become a target for security attacks. The modern approaches to IoT systems certifications still rely on traditional schemes like Common Criteria, the same scheme used for static web services; the issue with standard approaches is that they rely heavily on the staticity of the certified system and do not contemplate changes in the system nor in the context once deployed. Continuously dealing with complete recertifications every time the system undergoes some change is not feasible for the manufacturers for two main reasons: i) the certification process is long, it can last for months and could be triggered by a security patch, leaving the system exposed to a vulnerability for a long time and ii) the process is costly, easily falling into the hundreds of thousands of Euros for high-level certificates. 

The most effective way of developing assurance techniques is often by evolving the current standards, and so it has been, in the certification scheme field, until the IoT paradigm faced researchers with the abovementioned issues. Unfortunately, few efforts have been proposed to overcome the problem, such as [19][20] and [21], but they all rely on the traditional schemes and result in small variations of them, equally incapable of dealing with the dynamic nature of the IoT systems. Hence, the gap in the research is presented as a possible different direction that needs to be taken. Furthermore, IoT systems need space for changes without needing re-evaluation regularly and penalising their functionalities during an evaluation process. Therefore, this thesis is willing to take a first step toward solving the challenges posed by dynamic and pervasive architectures by tackling the following points: i) the certification process is made up of components that never really changed over the years, but in the IoT environment, they need to, ii) the certification schemes usually rely on the complete knowledge of the system, allowing them to precisely craft well-aimed tests over the features of the systems; this is not possible in a system involving thousands of devices and iii) traditional schemes never had to deal with limitations over systems' resources for their processes, with IoT they have to, in order not to clog the entire service. Thus, the goal of the approach proposed in this thesis is to finally have a faster, cheaper, lighter and more automated certification process for highly dynamic systems.

A certification scheme that allows reducing the weight over the system's components through automation and alternative testing techniques, drastically reduces the redundancy, the time and the resources needed to obtain a certificate; such reductions are fundamental for a system that possibly needs to undergo a high number of certification processes.

The methodology used in this thesis to approach our goal mainly focuses on the formal definition of the components that constitute the final certification process. First off, two new elements are added to the traditional set of certification components, the scoring system and the trigger. The scoring system is a feature that allows guiding the process using numerical values (e.g. scores and score thresholds) to evaluate the impact of each system’s feature in terms of value and effort needed to test it; the trigger is a component whose task is monitoring the system and detecting changes, eventually triggering the certification process and providing the necessary information to complete it. Then, we introduce the new models of the remaining components: i) attributes, ii) properties, iii) evidence collection and iv) certificate; such components needed to be revisited to fit the scheme correctly.

\newpage
The thesis work is structured as follows:
\begin{description}
    \item[State of the Art] Study of the current efforts toward the certification schemes in dynamic systems to better visualize the background and context considered in this thesis;
    \item[Methodology] Introduction of the methodology of the proposed certification scheme over a simple simulated scenario, where the certification components' models and phases are introduced, and the process is simulated, step-by-step, in an IoT smart-city example;
    \item[Experiments] Execution of the proposed certification scheme over a real-world-inspired scenario, where each process step is shown and accurately described, finally leading to the certificate's release.
\end{description}



\chapter{State Of The Art}
\label{cap2}
This chapter introduces the main system paradigms that today guide the various security certification frameworks.

\section{Cloud Computing Paradigm}
The National Institute of Standards and Technology (NIST) defined the Cloud computing paradigm as follows: ``Cloud computing is a model for enabling ubiquitous, convenient, on-demand network access to a shared pool of configurable computing resources (e.g., networks, servers, storage, applications, and services) that can be rapidly provisioned and released with minimal management effort or service provider interaction \cite{mell2011nist}."

The most commonly used deployment model for Cloud computing is the public Cloud, which allows users to access its resources through the Internet and is generally subject to monetization from provider companies.
Another commonly used deployment model is the private Cloud, usually found in single organizations for a more secure digital environment.
The other two models are Hybrid Cloud and Community Cloud, where the former is a mixture of public and private Cloud that overcomes some of the limitations of each, and the latter is an expansion from a private cloud, allowing multiple organizations to access its resources \cite{atlam2017integration}.

The essential services that Cloud computing offers include infrastructure as a service (IaaS), platform as a service (PaaS) and software as a service (SaaS), each of whom allows a different calibre of resources' control between the user and the provider \cite{khan2019edge}.

%%%%%%%%%%%%%%%%%%%%%%%%%%%%%%%%%%%%%%%%%%%%%%%%%%%%%%%%%%%%%%%%%%%%%%%%%%%%%%

\subsection{IoT in Cloud Environments}
Given the general flexibility, dynamicity and resource availability of Cloud computing, it has been the first adopted solution for IoT (Internet of Things) systems implementations.
The IoT label defines the network created around small smart devices interconnected through the Internet. To be classified as an IoT device, a piece of hardware has to be embedded with electronics, software, and connectivity in a way that enables it to communicate with other devices and exchange data that can be directly gathered from the device itself thanks to specific embedded sensors or from other connected devices.
The range of possible objects that today are intended as Smart IoT devices is vast; it comprehends instruments, vehicles, buildings, all sorts of sensor-enabled devices \cite{gokhale2018introduction}, industrial machines, and small objects like clothing, packages, parts, materials and much more. All these objects are active participants in the network and thus can be monitored, tracked and counted.
Cloud computing and IoT are built with complementary ideas; on the one hand, the Cloud is ubiquitous, secure, flexible and equipped with such a broad resource capability that it is often referred to as infinite. However, on the other hand, IoT devices are distributed and capable of generating immense moles of data that need a powerful computing element to analyze and process them.
Cloud computing simplifies the IoT data flow significantly and helps overcome many devices' limitations such as security, privacy, performance, and reliability. The main benefits of IoT integration in Cloud environments are widely discussed in \cite{atlam2017integration}.

As mentioned above, IoT devices generate a significant constant flow of data that for sure cannot be handled by the small devices themselves, but that is also becoming an issue for the Cloud infrastructures since IoT systems grow bigger and bigger every year, with more sensors and more communicating units [need source].
The biggest challenge in the IoT field is managing the massive quantity of data generated, and even the Cloud solutions are challenged by the large scale, heterogeneity and high latency issues. One rapidly increasing solution in popularity consists of using a decentralized computing model known as Fog Computing \cite{iorga2018fog}.

%%%%%%%%%%%%%%%%%%%%%%%%%%%%%%%%%%%%%%%%%%%%%%%%%%%%%%%%%%%%%%%%%%%%%%%%%%%%%%

\section{Edge Computing Paradigm}
\label{Edge}
Edge computing directs computational data, applications, and services away from Cloud servers to the edge of a network (figure \ref{fig:ceiot_stack}). The content providers and application developers can use the Edge computing systems by offering the users services closer to them. Edge computing is characterized in terms of high bandwidth, ultra-low latency, and real-time access to the network information that can be used by several applications \cite{khan2019edge}.

Edge or Fog computing is the latest and most promising researched solution to the recent trends of distributing the computation closer to the data sources. The goal is to provide low latency, high capacity and network efficient computation to IoT systems bringing multiple small Cloud nodes (Fog nodes) between the Cloud and the IoT devices at the ``edge" of the network. As for many concepts, the definitions are numerous \cite{sahni2018data} but Yi et al. stated the following as a possible definition: ``Fog Computing is a geographically distributed computing architecture with a resource pool which consists of one or more ubiquitously connected heterogeneous devices (including edge devices) at the edge of the network and not exclusively seamlessly backed by Cloud services, to collaboratively provide elastic computation, storage and communication (and many other new services and tasks) in isolated environments to a large scale of clients in proximity" \cite{yi2015fog}.
Practically Fog computing represents an extension of Cloud computing that offers multiple benefits to the latter:

\begin{itemize}
    \item Location awareness and low latency
    \item Geographical distribution
    \item Scalability
    \item Support for mobility
    \item Real-time interactions
    \item Heterogeneity
    \item Interoperability
    \item Support for online analytics and interplay with the Cloud
\end{itemize}

\begin{figure}[ht]
    \centering
    \fbox{\includegraphics[scale=0.55]{images/cloud_edge_iot_stack.png}}
    \caption{Example of cloud-edge-IoT stack architecture}
    \label{fig:ceiot_stack}
\end{figure}

\subsection{Edge and Cloud Relationship}
Edge computing is considered an extension of Cloud computing and not a standalone computing paradigm; it still relies on the Cloud at the top level of its scheme, but it drastically reduces the load over the Cloud infrastructures while maintaining all of the Cloud's provided services like data computing, storage and applications. The main difference between the two is the location of the servers; with the Cloud alone, latency dependant applications may encounter latency and jitter issues due to the distance between the user device and the server. On the other hand, Edge computing enables location-aware and mobile applications with full support, while Cloud applications need to find other workarounds. The longer path to the server can also be a weak point for security attacks on the Cloud. Another difference is the target audience: Cloud computing is a global solution, Edge computing is limited. Lastly, the single Edge node hardware is designed to be horizontally scalable through distribution, while Cloud computing is generally more vertically scalable \cite{khan2019edge}.


Today, an ever-rising number of applications must use software that satisfies strict reliability, availability and integrity requirements, especially when dealing with critical safety systems; hence, multiple assurance and verification techniques have been introduced, such as certification schemes, testing, service level agreements, audit/compliance and monitoring frameworks \cite{ardagna2015security}. 


\section{Security and Assurance in Distributed Systems}
Software security assurance techniques enhance software and services transparency \cite{ardagna2014management} and increase the actors' confidence that such services behave as expected. In line with standard software security assurance definitions \cite{goertzel2007software}, cloud security assurance can be defined as the way to gain justifiable confidence that infrastructure and applications will consistently demonstrate one or more security properties and operationally behave as expected despite failures and attacks. However, assurance is a much wider notion than security, as it includes methodologies for collecting and validating evidence supporting security properties. When dealing with cloud infrastructure, Ardagna et al. \cite{article} highlighted three requirements that need to be considered with every assurance technique: i) risk assessment and management, ii) transparency and iii) public policy and compliance. 

Risk assessment and management are needed before the cloud service deployment and allow the business to obtain a precise evaluation of the risks that such a process would introduce. Transparency is a core component when defining and optimizing assurance techniques and allows end-users to visualize and control the data flow and the security issues in their services. In addition, transparency is necessary to support introspection, the capability of a cloud provider to examine and observe its internal processes, and outrospection, which is the same thing but from the customer's point of view \cite{ardagna2014management}. To comply with the transparency requirement, cloud providers must show their policies and compliance with standards and regulations and how such compliance is achieved \cite{macneil2006comply}. The major approaches to cloud assurance are discussed below.
\begin{description}
    \item[Testing]
    Testing involves executing software or system services with the help of manual or automated tools to evaluate specific behaviours and properties. The test activity allows gathering evidence to support the evaluation target's property; properties can belong to the cloud infrastructure or the service running on it.

    \item[Monitoring]
    Monitoring processes can be deployed to overcome the limits of direct testing approaches; it allows for gathering precise information about the status of services, events, and activities on the back end. In addition, the security of a cloud system can be greatly improved with monitoring approaches since they also increase transparency.

    \item[Certification]
    Various certification techniques have been developed to prove that a software system holds some non-functional properties and behaves as expected. The advantage of such techniques is the presence of a certificate released at the end of a successful process; such a certificate can then be shown as undisputable proof.
    
    \item[Audit and Compliance]
    An essential aspect of security and assurance is the capability of observing the system's behaviour and evaluating its compliance with customer policies and law regulations. Audit solutions have been developed with this very goal; additionally, they increase the system's transparency making systems like the cloud more accountable in case of issues and failures.
    
    \item[Service-Level Agreement (SLA)]
    SLA-based techniques establish contracts between clients and service providers, regulate their interactions, and model their expectations in functional and non-functional agreements \cite{article}.

\end{description}






\subsection{Security Certifications}
Security certifications are granted by independent (from the system supplier and acquirer) third parties whose goal is to verify the system's quality completely and objectively. Certifications allow more confidence in the product over the competition and legislative compliance when talking about quality, functionality and security \cite{heck2010software}. The certification schemes' main focus is to prove that the selected software system or service would have some non-functional properties and behave as expected. 
Over the years, there have been proposed two main types of certification approaches: i) static, such as the ones proposed by \cite{anisetti2013test}\cite{CSATrustSTAR} ii) continuous and incremental, such as Common Criteria \cite{anisetti2017semi}\cite{infrastructure2002common}. 

A generic software certification process requires two types of input: i) The software artefact that needs to be certified and ii) one or more Conformance Properties of the artefact. The properties usually fall into one of the following categories:

\begin{itemize}
    \item Consistency
    \item Functional
    \item Behavioural
    \item Quality
    \item Compliance 
\end{itemize}

Although in the most recent years, the NIST proposed to definitely classify the systems' properties under just the Confidentiality, Integrity and Availability categories, more about this in the [properties] chapter. Moreover, the properties can be generic or dedicated to the specific artefact and need to be appropriate to the application's domain; this is established in a Conformance Analysis process \cite{anisetti2017semi}.


\subsection{Certification Schemes History}
It is quite difficult to retrace the full history of software security certification schemes due to the vastity of the software products' types and the fact that many countries started developing their certification schemes when such products needed some form of assurance. For these reasons, software products requiring security properties resulted in having a rather steep way toward international distribution. Therefore, the first approaches started dealing with the main pillars of certification schemes, such as properties, models and evidence collection (Fig. \ref{Fig:OldProcess}). Starting from the history of the most known scheme, the Common Criteria, the first attempts at defining a standard approach to such a complex task were made between 1983 and 1993 with the Information Technology Security Evaluation Criteria (ITSEC), the Canadian Trusted Computer Product Evaluation Criteria (CTCPEC) and the Trusted Computer System Evaluation Criteria (TCSEC).
\begin{figure}[htb]
\includegraphics[width=\textwidth]{immagine_web_service.png}
\caption{Conceptual framework (a) and certification process steps (b) \cite{anisetti2013test}}
\label{Fig:OldProcess}
\end{figure}
\subsubsection{ITSEC}
ITSEC is a structured set of criteria with the scope of evaluating security within computer systems and products in the European Union; it was published in 1990 and was developed by France, Germany, Netherlands and UK governments, basing its features on the pre-existing, country-limited frameworks. This set of criteria was designed in the main part to be equally applicable to technical security measures in hardware, software and firmware products \cite{ITSEC}. ITSEC was centred around the Target of Evaluation (ToE) and assurance levels; these concepts made their way through all the versions of today's Common Criteria \cite{infrastructure2002common}.

\subsubsection{TCSEC}
TCSEC, commonly referred to as ``The Orange Book" \cite{orangeBook}\cite{orangeBookDeath}, is a standard developed by the United States Government Department of Defense, and its scope was to evaluate and classify computer systems used for processing, storing and retrieving sensitive or classified information. This framework, like many others \cite{infrastructure2002common}\cite{ITSEC}, was developed around the following Evaluation Classes: 
\begin{itemize}
    \item D: Minimal Protection
    \item C1: Discretionary Security Protection
    \item C2: Controlled Access Protection
    \item B1: Labeled Security Protection
    \item B2: Structured Protection
    \item B3: Security Domains
    \item A1: Verified Design
    \item A2: Verified Implementation
\end{itemize}

\subsubsection{CTCPEC}
CTCPEC is a standard developed by the Canadian government with the same goals as ITSEC and TCSEC; in fact, it was a combination of the two. This approach finally led to combining the major standards and criteria into a single set: the Common Criteria \cite{CTCPEC}.

\subsubsection{Common Criteria}
\label{CC}
With these internationally approved standards, software companies finally had the tools to certify their products and sell them worldwide; however, every certification process had to be passed, resulting in a rather tedious and pricey process. Therefore, the Common Criteria (CC) was developed by unifying these pre-existing standards, following the approach of the CTCPEC, to allow companies to evaluate their products against a single set of standards. CC was developed by Canada, France, Germany, Netherlands, UK and USA governments and version 1.0 was issued in 1994; multiple agreements were signed in the following years to avoid useless re-evaluations and allow mutual recognition of CC certificates.
The Common Criteria for Information Technology Security Evaluation (also known as Common Criteria) represents the evolution of continuous and incremental certification schemes, and its latest versions are currently used to evaluate over two thousand major IT systems and applications (e.g. Microsoft Windows OS, McAfee anti-virus, Microsoft SQL Server \cite{CCProducts}).
CC is mainly based on the previous sets of standards' ideas:
\begin{itemize}
    \item Definition of the ToE, which includes the assignment of a Protection Profile, the description of the Security Target and the listing of the Security Functional Requirements.
    \item Definition of the Security Assurance Requirements
    \item Selection of the correct Evaluation Assurance Level
\end{itemize}

More details about the above phases are well described in \cite{infrastructure2002common}.

CC is now an established pillar in the security certification world, especially in the commercial industry, but it is not a perfect solution, especially when applied to highly dynamic systems like Edge and IoT. It is also worth noting that CC certifications are attributed to a specific version of the system under a specific configuration, and any configuration change or software update could mandate a re-certification process. To counter the redundancy of such processes, the CC designers proposed the Assurance Continuity (CCAC) re-evaluation process[], but the need to repeat the certification process is a big limitation. Therefore, the research never stopped; when considering highly dynamic domains such as those implied by the Cloud and Fog computing paradigms, it is important to define new certification schemes that better align with the context. 

%%%%%%%%%%%%%% da decidere se lasciarlo qui %%%%%%%%%%%%%%%%%%%%%%%%%


\section{Traditional Certification Frameworks Composition}
Modern certification frameworks rely upon several components defining how the process is executed and how the results should look. Below is a detailed list of the major pillars of certification frameworks with their fundamental functionalities as underlined by Anisetti et al. [anisetti2022multi][anisetti2017semi].

\subsection{Non-Functional Properties}
For the certification process of a system, it is important to establish some requirements that can be functional or non-functional preemptively, usually this is done during an initial risk assessment process. As stated in [TROVA UN ALTRA SOURCE QUESTO E DA WIKIPEDIA]: "broadly, functional requirements define what a system is supposed to do, and non-functional requirements define how a system is supposed to be", meaning that non-functional requirements focus on the system itself instead of what the system produces in output; once a requirement is proven to be satisfied by the system, it becomes a non-functional property of the same, and it can be expressed as follows: ⟨name, {(attribute, value), (attribute, value)}⟩. A property is composed of pairs "attribute-value" where the attributes are essential sub-properties that define the strength of the property.

In other words, non-functional properties describe the system's capabilities and are usually grouped under the categories defined by the CIA triad: i) Confidentiality, ii) Integrity and iii) Availability. The CIA triad is an organizational model designed to guide information storing policies; its three parts are cybersecurity's three most crucial components.

These properties are considered abstract because they represent generic security requirements and provide no information on how to achieve them. Instead, they are used to define concrete security properties. 

As defined by Anisetti et al. in [web services], a concrete security property is formally defined as follows:
A concrete security property, denoted p, is a pair (pˆ,A)\footnote{TUTTE LE FORMULE NELLO STATO DELL ARTE SONO DA SISTEMARE}, where p.pˆ is an abstract property and p.A is a set of class attributes specifying the threats the service proves to counteract or the specific characteristics of the security function implemented by the service.

The NIST defined these components as follows:
\begin{description}
    \item[Confidentiality] Preserving authorized restrictions on information access and disclosure, including means for protecting personal privacy and proprietary information. (NIST)
    \item[Integrity] Guarding against improper information modification or destruction, and includes ensuring information non-repudiation and authenticity.(NIST)
    \item[Availability] Ensuring timely and reliable access to and use of information.(NIST)

\end{description}

Confidentiality is a set of rules that limits access to information; ensuring confidentiality means the authorized entities can access the protected information and the unauthorized entities cannot. Confidentiality is accomplished through methods like data encryption and authentication procedures.

Integrity is the assurance that the information is trustworthy and accurate; it involves maintaining data consistency and accuracy over its entire lifecycle. For example, measures like file permissions and user access control ensure unauthorized entities cannot change data, while checksums and backups safeguard data from non-human threats like electromagnetic pulses or server crashes.

Availability guarantees reliable access to the information; it is best ensured by rigorously maintaining all hardware and staying up to date with all system upgrades; providing bandwidth, preventing bottlenecks and fast disaster recovery are also essential.

The NIST officially published several technical and non-technical capabilities that an IoT system’s manufacturer should consider; It should be noted that the following security requirements have been officially classified under one of the three parts of the CIA triad by the NIST.

Device identification, which is the capability to identify the IoT device for multiple purposes and in multiple ways to meet organizational requirements, is composed of the following requirements:

\begin{description}
    \item[Identifier Management Support] Ability for device identification. Elements that may be necessary:
        \begin{itemize}
            \item Ability to uniquely identify the IoT device logically.
            \item Ability to uniquely identify a remote IoT device.
            \item Ability for the device to support a unique device ID (e.g., to allow it to be linked to the person or process assigned to use the IoT device).
        \end{itemize}
    \item[Device Authentication Support] Ability to support local or interfaced device authentication. Elements that may be necessary:
        \begin{itemize}
            \item Ability for the IoT device to identify itself as an authorized entity to other devices.
            \item Ability to verify the identity of an IoT device.
        \end{itemize}
    \item[Physical Identifiers] Ability to add a unique physical identifier at an external or internal location on the device authorized entities can access.
\end{description}

On the other hand, the device configuration is defined as The capability to configure the IoT device through logical or physical interfaces to meet organizational requirements and includes the following requirements:
\begin{description}
    \item[Logical Access Privilege Configuration] Ability for only authorized entities to apply logical access privilege settings within the IoT device and configure logical access privilege as described in Logical Access to Interfaces.
    \item[Authentication and Authorization Configuration] Ability for only authorized entities to configure IoT device authentication policies and limitations as described in Logical Access to Interfaces.
    \item[Interface Configuration] Ability for only authorized entities to configure aspects related to the device’s interfaces as described in Logical Access to Interfaces.
    \item[Display Configuration] Ability to configure content to be displayed on a device.
    \item[Device Configuration Control] Ability to change configurations on the IoT device based on operational events as described in Device Security and Cybersecurity Event Awareness.
    \begin{itemize}
        \item Ability to change the device’s software configuration settings.
        \item Ability for authorized entities to restore the device to a secure configuration defined by an authorized entity.
        \item Configuration settings for use with the Device Configuration capability including, but not limited to:
        \begin{itemize}
            \item Ability for authorized entities to configure the cryptography use itself, such as choosing a key length.
            \item Ability to configure any remote update mechanisms to be either automatically or manually initiated for update downloads and installations.
            \item Ability to enable or disable notification when an update is available and specify who or what is to be notified.
        \end{itemize}
    \end{itemize}
\end{description}

Every IoT system revolves around data, and protecting it is the core focus of cybersecurity; the NIST defines data protection as the capability to protect IoT device data to meet organizational requirements. Therefore, it is important to satisfy as many of the following requirements as possible to achieve a high level of data protection.

\begin{description}
    \item[Cryptography Capabilities and Support] Ability for the IoT device to use cryptography for data protection. Elements that may be necessary:
        \begin{itemize}
            \item Ability to execute cryptographic mechanisms of appropriate strength and performance.
            \item Ability to obtain and validate certificates.
            \item Ability to verify digital signatures.
            \item Ability to run hashing algorithms.
            \item Ability to perform authenticated encryption algorithms.
            \item Ability to compute and compare hashes.
        \end{itemize}
    \item[Cryptographic Key Management] Ability to manage cryptographic keys securely:
        \begin{itemize}
            \item Ability to generate key pairs.
            \item Ability to store encryption keys securely.
            \item Ability to change keys securely.
        \end{itemize}
    \item[Secure Storage] Ability for the IoT device, or tools used through the IoT device interface, to enable secure device storage. Elements that may be necessary:
        \begin{itemize}
            \item Ability to support encryption of data at rest.
            \begin{itemize}
                \item Ability to cryptographically store passwords at rest, as well as device identity and other authentication data.
                \item Ability to support data encryption and signing to prevent data from being altered in device storage.
            \end{itemize}
            \item Ability to secure data in device storage
                \begin{itemize}
                    \item Ability to secure data stored locally on the device.
                    \item Ability to secure data stored in remote storage areas (e.g., cloud, server, etc.).
                    \item Ability to utilize separate storage partitions for system and user data.
                \end{itemize}
            \item Ability to “sanitize” or “purge” specific or all data in the device.
        \end{itemize}
        
    \item[Secure Transmission] Ability to secure data transmissions sent to and from the IoT device. Elements that may be necessary:
    \begin{itemize}
        \item Ability to configure the cryptographic algorithm to protect data in transit.
        \begin{itemize}
            \item Ability to support trusted data exchange with a specified minimum strength cryptography algorithm.
            \item Ability to support data encryption and signing to prevent data from being altered in transit.
        \end{itemize}
        \item Ability to utilize one or more capabilities to protect the data it transmits from unauthorized access and modification.
        \item Ability to use cryptographic means to validate the integrity of data transmitted.
    \end{itemize}
\end{description}

Another important aspect is the security of logical access to the system's interfaces, which is the ability to require authentication to or identify the IoT device and to establish authentication and identification configuration and display requirements.

\begin{description}
    \item[Authentication Support] Ability to support authentication methods.
    \begin{itemize}
        \item Ability for the IoT device to require authentication prior to connecting to the device.
        \item Ability for the IoT device to support and require appropriate authentication
        \begin{itemize}
            \item Ability for the IoT device to require authentication prior to connecting to the device.
            \item Ability for the IoT device to support a second, or more, authentication method(s) through an out of band path such as temporary passwords or other one-use logon credentials, third-party credential checks, biometrics, text messages, hard tokens and manufacturer proprietary methods
        \end{itemize}
        \item Ability for the IoT device to hide or mask authentication information during authentication process.
        \item Ability for the IoT device to support a second, or more, authentication method(s) through an out-of-band path such as: Temporary passwords or other one-use credentials; Third-party credential checks; Biometrics; Text messages; Hard Tokens; etc.
    \end{itemize}
    
    \item[Authentication Configuration] Ability to require, or not require, authentication to, and/or identification of, the IoT device, and to establish authentication and identification configuration and display requirements. Elements that may be necessary:
    \begin{itemize}
        \item Ability to set and change authentication configurations, policies and limitations settings for the IoT device
        \begin{itemize}
            \item Ability to set the time period for how long the device will remain locked after an established configurable limit of unsuccessful login attempts has been met.
            \item Ability to disable or lock access to the device after an established number of unsuccessful login attempts.
            \item Ability to display and/or report the previous date and time of the last successful login authentication.
            \item Ability to automatically disable accounts for the IoT device after an establish period of inactivity
            \item Ability to support automatic logout of inactive accounts after a configurable established time period.
            \item Ability to support automatic removal of temporary, emergency and other special use accounts after an established time period.
        \end{itemize}
        \item Ability to authenticate external users and systems
        \item Ability to revoke their access.
    \end{itemize}
    
    \item[System Use Notification Support] Ability to support system use notifications.
    \begin{itemize}
        \item Ability to display to IoT device users an organizationally-defined system use notification message or banner prior to successful IoT device authentication. (e.g., the message or banner would provide privacy and security notices consistent with applicable federal laws, Executive Orders, directives, policies, regulations, standards, and guidance).
        \item Ability to create an organizationally-defined system use notification message or banner to be displayed on the IoT device
        \begin{itemize}
            \item Ability to edit an existing IoT device display.
            \item Ability to establish the maximum size (in characters, bytes, etc.) of the available device display.
        \end{itemize}
        \item Ability to keep the notification message or banner on the device screen until the device user actively acknowledges and agrees to the usage conditions
    \end{itemize}
    
    \item[Authorization Support] Ability to restrict all unauthorized interactions.
    \begin{itemize}
        \item Ability to identify authorized users and processes.
        \item Ability to differentiate between authorized and unauthorized users (physical and remote).
    \end{itemize}
    
    \item[Authentication And Identity Management] Ability to establish access to the IoT device to perform organizationally-defined user actions without identification or authentication.
    
    
    \item[Role Support And Management] Ability to establish unique, privileged, organization-wide, and other types of IoT device user accounts. Elements that may be necessary:
    \begin{itemize}
        \item Ability to create unique IoT device user accounts.
        \item Ability to assign roles to IoT device user accounts.
        \item Ability to identify unique IoT device user accounts.
        \item Ability to support a hierarchy of logical access privileges for the IoT device based on roles (e.g., admin, emergency, user, local, temporary, etc.).
        \begin{itemize}
            \item Ability to establish user accounts to support role-based logical access privileges.
            \item Ability to administer user accounts to support role-based logical access privileges.
            \item Ability to use organizationally-defined roles to define each user account’s access and permitted device actions.
            \item Ability to support multiple levels of user/process account functionality and roles for the IoT device.
        \end{itemize}
        \item Ability to apply least privilege to user accounts (i.e., to ensure that the processes operate at privilege levels no higher than necessary to accomplish required functions).
        \begin{itemize}
            \item Ability to create additional processes, roles (e.g., admin, emergency, temporary, etc.) and accounts as necessary to achieve least privilege.
            \item Ability to apply least privilege settings within the device (i.e., to ensure that the processes operate at privilege levels no higher than necessary to accomplish required functions).
            \item Ability to limit access to privileged device settings that are used to establish and administer authorization requirements.
            \item Ability for authorized users to access privileged settings.
        \end{itemize}
        \item Ability to support organizationally-defined actions for the IoT device.
        \begin{itemize}
            \item Ability to create organizationally-defined accounts that support privileged roles with automated expiration conditions.
            \item Ability to establish organizationally-defined user actions for accessing the IoT device and/or device interface.
            \item Ability to enable automation and reporting of account management activities.
            \item Ability to assign access to IoT device audit controls to specific roles or organizationally-defined personnel.
            \item Ability to control access to IoT device audit data.
            \item Ability to identify the user, process or device requesting access to the audit/accountability information (i.e., to ensure only authorized users and/or devices have access).
            \item Ability to establish conditions for shared/group accounts on the IoT device.
            \item Ability to administer conditions for shared/group accounts on the IoT device.
            \item Ability to restrict the use of shared/group accounts on the IoT device according to organizationally-defined conditions.
        \end{itemize}
        \item Ability to implement dynamic access control approaches (e.g., service-oriented architectures) that rely on:
        \begin{itemize}
            \item run-time access control decisions facilitated by dynamic privilege management.
            \item organizationally-defined actions to access/use device.
        \end{itemize}
        \item Ability to allow information sharing capabilities based upon the type and/or role of user attempting to share the information.
        \item Ability to restrict access to IoT device software, hardware, and data based on user account roles, used with proper authentication of the identity of the user to determine type of authorization.
    \end{itemize}
    
    
    \item[Limitations on Device Usage] Ability to establish restrictions for how the device can be used. Elements that may be necessary:
    \begin{itemize}
        \item Ability to establish pre-defined restrictions for information searches within the device.
        \item Ability to establish limits on authorized concurrent device sessions for:
        \begin{itemize}
            \item User accounts
            \item Roles
            \item Groups
            \item Dates
            \item Times
            \item Locations
            \item Manufacturer established parameters
        \end{itemize}
    \end{itemize}
    
    
    \item[External Connections] Ability to support external connections. Elements that may be necessary:
    \begin{itemize}
        \item Ability to securely interact with authorized external, third-party systems.
        \item Ability to allow for the user/organization to establish the circumstances for when information sharing from the device and/or through the device interface will be allowed and prohibited.
        \item Ability to establish automated information sharing to approved identified parties/entities.
        \item Ability to identify when the external system meets the required security requirements for a connection.
        \item Ability to establish secure communications with internal systems when the device is operating on external networks.
    \end{itemize}
    
    \item[Interface Control] Ability to establish controls for the connections made to the IoT device. Elements that may be necessary:
    \begin{itemize}
        \item Ability to establish requirements for remote access to the IoT device and/or IoT device interface including:
        \begin{itemize}
            \item Usage restrictions
            \item Configuration requirements
            \item Connection requirements
            \item Manufacturer established requirement
        \end{itemize}
        \item Ability to restrict use of IoT device components (e.g., ports, functions, microphones, video).
        \item Ability to logically or physically disable any local and network interfaces that are not necessary for the core functionality of the device.
        \item Ability to restrict updating actions to authorized entities.
        \item Ability to restrict access to the cybersecurity state indicator to authorized entities.
        \item Ability to restrict use of IoT device services.
        \item Ability to enforce the established local and remote access requirements.
        \item Ability to prevent external access to the IoT device management interface.
        \item Ability to control the IoT device’s logical interface (e.g., locally or remotely).
        \item Ability to change IoT device logical interface(s).
        \item Ability to control device responses to device input.
        \item Ability to control output from the device.
        \item Ability to support wireless technologies needed by the organization (e.g., Microwave, Packet radio (UHF/VHF), Bluetooth, Manufacturer defined)
        \item Ability to support communications technologies (including but not limited to):
        \begin{itemize}
            \item IEEE 802.11
            \item Bluetooth
            \item Ethernet
            \item Manufacturer defined
        \end{itemize}
        \item Ability to establish and configure IoT device settings for wireless technologies including authentication protocols (e.g., EAP/TLS, PEAP).
    \end{itemize}
\end{description}
    
    Given the intrinsic dynamicity of IoT systems and devices, software updates are frequent, and support mechanisms may need additional security properties. These requirements refer to the ability to update IoT device software and support mechanisms for such updates.
    \begin{description}
        \item[Update Capabilities] Ability to update the IoT device software within the device and/or through the IoT device interface. Elements that may be necessary:
        \begin{itemize}
            \item Ability to update the software by authorized entities only using a secure and configurable mechanism.
            \item Ability to identify the current version of the organizational audit policies and procedures governing the software update.
            \item Ability to restrict software installations to only authorized individuals or processes.
            \item Ability to restrict software changes/uninstallations to only authorized individuals or processes.
            \item Ability to verify software updates come from valid sources using an effective method (e.g., digital signatures, checksums, certificate validation, etc).
        \end{itemize}
        
        \item[Update Application Support] Ability to update the device’s software through remote (e.g., network download) and/or local (e.g., removable media) means
        \begin{itemize}
            \item If software updates are delivered and applied automatically:
            \begin{itemize}
                \item Ability to verify and authenticate any update before installing it
                \item Ability to enable or disable updating
            \end{itemize}
        \end{itemize}
    \end{description}
    
    
    %% skipping cybersecurity awareness %%
    
    Last but not least important is the device's security, defined as the capability to secure the IoT device to meet organizational requirements.
    
    \begin{description}
        \item[Secure Execution] Ability to protect the execution of code on the device. Elements that may be necessary:
        \begin{itemize}
            \item Ability to enforce organizationally-defined execution policies.
            \begin{itemize}
                \item Ability to execute code in confined virtual environments.
                \item Ability to separate IoT device processes into separate execution domains.
            \end{itemize}
            \item Ability to separate the levels of IoT device user functionality.
            \item Ability to authorize various levels of IoT device functionality.
        \end{itemize}
        
        
        \item[Secure Communication] Ability to securely initiate and terminate communications with other devices. Elements that may be necessary:
        \begin{itemize}
            \item Ability to enforce traffic flow policies.
            \item Ability to utilize standardized protocols.
            \item Ability to establish network connections.
            \item Ability to terminate network connections (e.g., automatically based on organizationally-defined parameters).
            \item Ability to de-allocate TCP/IP address/port pairings.
            \item Ability to establish communications channels.
            \item Ability to secure the communications channels.
            \item Ability to interface with DNS/DNSSEC.
            \item Ability to store and process session identifiers.
            \item Ability to identify and track sessions with identifiers.
        \end{itemize}
        
        \item[Secure Resource Usage] Ability to securely utilize system resources and memory. Elements that may be necessary:
        \begin{itemize}
            \item Ability to support shared system resources.
            \begin{itemize}
                \item Ability to release resources back to the system.
                \item Ability to separate user and process resources use.
                \item Ability to no one will read this.
            \end{itemize}
            \item Ability to manage memory address space assigned to processes.
            \item Ability to enforce access to memory space through the kernel.
            \item Ability to prevent a process from accessing memory space of another process.
            \item Ability to enforce configured disk quotas.
            \item Ability to continue operation when associated networks are unavailable (e.g., a smart smoke detector must still go off when a fire occurs even if it is not attached to the associated network).
            \item Ability to provide sufficient resources to store and run the operating environment (e.g., operating systems, firmware, applications).
            \item Ability to utilize file compression technologies (e.g., to provide denial of service protection).
        \end{itemize}
        
        
        \item[Device Integrity] Ability to protect against unauthorized changes to hardware and software. Elements that may be necessary:
        \begin{itemize}
            \item Ability to perform security compliance checks on system components.
            \item Ability to detect unauthorized hardware and software components.
            \item Ability to take organizationally-defined actions when unauthorized hardware and software components are detected (e.g., disallow a flash drive to be connected even if a USB port is present).
            \item Ability to store the operating environment (e.g., firmware image, software, applications) in read-only media (e.g., Read Only Memory).
        \end{itemize}
        
        \item[Secure Device Operation] Ability to operate securely and safely. Elements that may be necessary:
        \begin{itemize}
            \item Ability to keep an accurate internal system time.
            \item Ability to define various operational states.
            \item Ability to support various modes of IoT device operation with more restrictive operational states.
            \begin{itemize}
                \item “travel mode” for transit.
                \item “safe mode” for operation when some or all network security is unavailable.
                \item Others as determined necessary based on the purpose and goals for the IoT device.
            \end{itemize}
            \item Ability to define differing failure types.
            \item Ability to fail in a secure state.
            \item Ability to disable operations and/or functionality in the event of security violations.
            \item Ability to restrict components/features of the IoT device (e.g., ports, functions, protocols, services, etc.) in accordance with organizationally-defined policies.
            \item Ability to sense the environment and securely (i.e., preserving confidentiality, integrity, and availability of the device and its data) interface with the environment, either directly or through the IoT system. Examples include:
            \begin{itemize}
                \item Emergency shutoff mechanism
                \item Emergency lighting mechanism
                \item Fire protection mechanism
                \item Temperature and humidity mechanism
                \item Water damage protection mechanism
                \item Manufacturer defined capability
            \end{itemize}
        \end{itemize}
        
        
    \end{description}
    
    
    Moreover, Nasiri et al. \cite{nasiri2019security} defined a series of security requirements that manufacturers should consider (and develop) when designing IoT systems. These requirements are a high-level summary of the above that could be useful while showcasing examples of systems’ design. Although the research refers to healthcare IoT systems, many properties are still relevant for any kind of IoT system.
    
    \begin{description}
        \item[Identification and authentication] Identification guarantees the identity of all the entities before permitting them to interact with the resources of the IoT system. Authentication is the process of confirming the identity of a person or device before using the system resources. Devices and application authentication can prove that the interacting system is not an adversary and that data shared in networks is legal.
        
        \item[authorization (access control)] After user identity verification, access rights or privileges to resources should be determined so that different users can only access the resources required based on their tasks.

        \item[privacy] Privacy means that secrets and personal data of users should not be disclosed without consent. Therefore, the IoT system should be in accordance with privacy policies allowing users to control their private data.

        \item[accountability] In a health IoT system, accountability should ensure that the organization or individuals are obliged to be answerable or responsible for their actions in case of theft or abnormal event.

        \item[non-repudiation] Non-repudiation ensures that someone cannot deny an action that has already been done. It enables the users to prove the occurrence or non-occurrence of an event.

        \item[auditing] Auditing is the ability of a system to track and monitor actions continuously. All user activities should be recorded in sequential order, such as system login time and data modification.
        
        \item[data freshness] Data freshness means that data should be recent, ensuring that no old messages are replayed.

        

    \end{description}
    

\subsection{Non-Functional Mechanisms}
For a non-functional property to be proven, a non-functional mechanism, in the form of ⟨name, {(attribute, value), (attribute, value)}⟩, must be implemented to execute the certification process under specific configurations. Moreover, the property is what the process needs to verify, and the mechanism is the tool the system offers to test such property[10]; the verification is usually done through tests in the form of ⟨mechanism, (code to execute, expected results)⟩. Finally, it is important to note that the mechanisms need to be described in the system's documentation; this allows the accredited laboratory to analyze and test them during the manual review phases.


\subsection{Target Of Certification}
The application context and the system perimeter define the Target of the Certification (ToC); it describes the mechanisms on which the non-functional properties that need to be certified rely. Moreover, the evaluation of the certification scheme must verify the target’s (security) features. For example, in the Common Criteria process, the ToC (actually named Target of Evaluation) is accurately described in the Security Target Document.

\subsection{Evidence}
Evidence collection is the main step to proving that a given non-functional property is valid for a given ToC; it is important to note that evidence can be gathered by testing the service actively or passively. 
The property is finally assigned to the system when enough evidence is collected for a given requirement through well-aimed tests. 


\subsection{Life Cycle}
\begin{figure}[hb]
    \centering
    \includegraphics[scale=0.5]{images/lifeCycle.png}
    \caption{The cybersecurity certificate during the IoT device lifecycle.}
    \label{fig:lifecycle}
\end{figure}
As mentioned above, certificates are subject to invalidation, whether caused by a context change, a property change, or simply by the certificate's expiration; all certificates have a finite lifespan managed by the Certificate Authority (CA), which is usually also in charge of replacing it with a new one. However, the CA represents a bottleneck, and automation frameworks are needed when dealing with highly dynamic environments like Cloud and Edge. Therefore, the life cycle of a certificate is de facto a deterministic finite-state automaton whose state transitions are determined by specified conditions.

As mentioned above, certificates are subject to invalidation, whether caused by a context change, a property change, or simply by the certificate's expiration; all certificates have a finite lifespan managed by the Certificate Authority (CA), which is usually also in charge of replacing it with a new one. However, the CA represents a bottleneck, and automation frameworks are needed when dealing with highly dynamic environments like Cloud and Edge. Therefore, the life cycle of a certificate is de facto a deterministic finite-state automaton whose state transitions are determined by specified conditions.

As shown in figure \ref{fig:lifecycle}, the lifecycle of a certificate is tightly coupled with the lifecycle of the IoT system, which is usually divided into five phases: i) Manufacturing, ii) Bootstrapping, iii) Operation, iv) Updating and v) Decommissioning.

The manufacturing phase includes the production, programming and configuration of the system/device. This phase also includes the security certification process, which, if passed, assigns a brand new certificate to the system.

The bootstrapping phase includes the installation and configuration of the device for the specific domain in which it will operate. Moreover, context-specific information should be embedded in the cybersecurity label (this step is too often ignored). It is also important to note that the context plays a fundamental role in every IoT system; a given device could require different security levels based on the type of information it needs to handle, and this variation can happen multiple times in the lifespan of the device.

During the operation phase, the device/system provides its intended functionalities and should always be kept under a monitoring process. The monitoring is crucial to find new vulnerabilities and issues in the system. When a new vulnerability is found, and the manufacturer issues an update, rarely is a re-certification not issued due to the complexity of the analysis of the changes. Moreover, the security level and label are updated during a re-certification process.

The decommissioning phase represents the end of the lifecycle of the system and includes the certificate revocation. This phase is also crucial because the devices composing the system could have stored sensitive information, and this phase's sole purpose is to make such information inaccessible.

Typically, a certification process for Cloud services consists of a collaborative effort between four entities: i) a service provider who develops the service that needs to be certified, ii) a cloud provider supporting the service's certification, iii) a CA responsible for the definition of the requirements and the methodology and iv) an accredited lab to carry out the system evaluation. The problem of such an approach resides in the fact that continuous context changes can rapidly invalidate the assigned certificates, and without proper context testing techniques, it is necessary to re-certificate the whole service every time. The framework that M. Anisetti et al. proposed addressed this issue by constantly verifying the possessed certifications against the context changes to prevent unnecessary revocations and re-certifications.


\subsection{Labels}
Labelling schemes exist to straightforwardly inform the user about the security certification that has been executed on a device; such schemes should contain the following main groups of information:

\begin{itemize}
    \item Domain that was considered while performing the certification activity; the certification should be revoked the moment the device leaves it

    \item Level of assurance; states the security tests performed to certify the device. Usually, the level of assurance is associated with the protection profile of the specific domain

    \item  Information about the achievement of the certification, such as the responsible entity, the process and the validity period

\end{itemize}

Even though the importance of labels is mostly tied to commercial purposes [12], perhaps, they could also be read by the Certification Authorities to have a quicker overview of the system's mechanisms and capabilities. In state-of-the-art schemes, labels are ignored or poorly used, but they might play a fundamental role in future schemes.




%%%%%%%%%%%%%%%%%%%%%%%%%%%%%%%%%%%%%%%%%%%%%%%%%%%%%%%%%%%%%%%%%%%%%

\subsection{New Certification Efforts in Cloud Environments}
Cloud service certification has been a highly focused research point in recent years and is now considered a mature assurance feature. Moreover, as identified by M. Anisetti et al. in \cite{anisetti2017semi}: ``Compared to traditional service certification, cloud certification is: i) highly dynamic, it is affected by contextual changes at any layer of the cloud stack, ii) multi-layer, it can refer to services at different cloud layers; iii) intrinsically incremental, it requires continuous validity verification and incremental adaptation with the scope to minimize costly re-certification activities, and iv) trustworthy by delegation, it requires advanced trust models based on delegation to support cloud peculiarities". M. Anisetti et al. also proposed a continuous and incremental approach to Cloud service certification, providing a formal description of the audit-related evidence collection. Such an approach is founded on five main pillars: non-functional properties, non-functional mechanisms, Target of Certification (ToC), evidence collection and the life cycle of the certification. The proposed certification process can be simplified with the following steps: i) Non-functional properties definition, ii) ToC definition, which includes the description of the non-functional mechanisms offered by the system, iii) Tests definition, iv) Tests execution, which involves evidence collection and analysis and v) Certificate release.


This process involves different entities, such as the service provider, the cloud provider, the certification authority and an accredited laboratory; this way, a chain of trust is formed, and the responsibilities are evenly spread between the entities involved. It is also split into an issuing phase, where the process is responsible for all the evaluation activities, and a post-issuing phase, which continuously verifies the certificate's validity against eventual context changes \cite{anisetti2017semi}.

\subsection{New Certification Efforts in Edge Environments}
Certification schemes over Fog/Edge computing nodes are a recent and less mature topic; when dealing with a network composed of countless devices (the Fog nodes), it is fundamental to have a completely automated framework capable of handling the whole certification process for each node. The scheme proposed by Aslam et al. in \cite{aslam2020fonac} addressed the matter with automated and continuous monitoring and auditing of the Fog nodes; nevertheless, this total automation relies entirely on the backend provided by the Cloud infrastructure that acts as an inner CA for its Fog layer, inheriting all its vulnerabilities. The differences between this approach and the ones proposed for the Cloud are the requirements for the certifications. Fog nodes have limited functionalities and need to be quickly admitted to the Fog layer; meanwhile, there is a need to execute much stronger certification activities on the Cloud stack.

\subsection{Current IoT Certification Efforts}
Internet of Things is not a new concept, and so is not the need for dedicated certification schemes; The rapid spread of systems relying on IoT highlighted many limitations of the most known certification schemes.

\subsubsection{Common Criteria}
As described in section \ref{CC}, CC is the most used on every type of system (Cloud, Edge, IoT) thanks to its Mutual Recognition Agreements (MRA) and to the depth the certification process can reach. However, the industry and the research community have underlined three major limitations that make it hardly applicable in the IoT world \cite{kaluvuri2014quantitative}\cite{keblawi2006applying}:
\begin{enumerate}
    \item It takes too much time and effort, resulting in a slowdown of the system and the risk of overwhelming the single devices with testing requests.
    \item Systems classified with high EALs are attributed to highly complex documentation, making the comparison with other certificates difficult, hence, not complying with the harmonization challenge (more about this in the commercial section).
    \item Managing changes is almost non-existent, even with the CCAC version, limiting the number of updates the manufacturer can issue and drastically reducing system operativity during re-certification.
\end{enumerate}


\subsubsection{ICSA Labs IoT Security Testing Framework}
A newer framework is the one proposed by the ICSA Labs company. The ICSA Labs IoT Security Testing Framework is focused on IoT systems, more specifically, on the security of the single devices participating in an IoT environment. The approach is based on a periodic assessment and update of the certification criteria, and it includes frequent iterative updates, addressing the dynamic environment challenges and the evolving cybersecurity threats inherent to the IoT paradigm. This process is more focused on the iterativity of the assessments and less on the depth of the certification, leaving aspects such as the labelling ignored \cite{ICSAFw}.

CC remains the most used and studied certification scheme, while others like Commercial Product Assurance, Cybersecurity Assurance Program and the Certification de Sécurité de Premier Niveau fall into a commercial and country-specific side.


\subsubsection{Industrial And Commercial Purpose Schemes}
From the commercial and industrial point of view, several challenges have been highlighted in the definition and deployment of an IoT security certification framework; some of these are not part of this thesis' focus but are still relevant for a complete overview of what a certification framework should guarantee at a certain point; furthermore, table \ref{Tab:comparison} shows a quick comparison of the approaches discussed below.
\begin{enumerate}
    \item Standardization: Since the IoT landscape is still fragmented in approaches and standards, it is crucial to start aligning the efforts for a more homogeneous perspective.


    \item Harmonization: while there is the need to align the standards, there is also the need to allow different ones to coexist and be mutually accepted.


    \item Time, complexity and cost: Manual processes, direct testing and complex documentation guidelines of currently available approaches are heavy on resources like time and money, and the need to minimize this weight is crucial for a for-profit company, especially with IoT systems that need frequent re-certifications.


    \item Dynamic environment: IoT systems are often deployed in dynamic environments whose conditions might affect the security level of a device. Moreover, a device should be able to receive an update and change its or the system's configuration without invalidating the certificate completely.


    \item Influence of the context: The context in which an IoT system is deployed is not ignorable; indeed, it determines many of the security features the system will require. Moreover, the context and the system's purpose should be linked to determine the sensitivity of the managed data.


    \item Transparency to the end-user: A company must help the clients understand the positive aspects of the system's certification, especially when different approaches are not directly comparable. For this purpose, it is becoming a common rule to attribute security labels to the certificated devices; such labels should clearly and concisely represent the value of the certification's level. Labels' goal is ultimately to allow users to easily compare different levels of frameworks without knowing the approaches' details.


    \item Support for the certificate lifecycle: cybersecurity certificates carry validity periods that regulate and confirm the authenticity of the involved entities. It is important to ensure that the certification is supported and updated during its lifecycle and that all the certificated properties remain valid during this period.

\end{enumerate}


It should be noted that apart from the AC version of Common Criteria, no other approach considers a lighter process for re-certification processes.

Some available frameworks are adopted more frequently than others; in the following section, the most known and used in the industry will be presented and compared to highlight the major weaknesses that should be considered when developing new certification schemes \cite{surveyIOT}.

\subsubsection{Commercial Product Assurance}
The Commercial Product Assurance (CPA) is a certification scheme developed by the UK Communications-Electronics Security Group (CESG) for national public purposes. CPA allows only security-related patches and updates, and thanks to its ``lightweight" design, it drastically reduced the re-certification process length; however, it is not enough. The average re-certification still takes at least six months to complete, reducing the system operativity during this time.

\subsubsection{Cybersecurity Assurance Program}
Cybersecurity Assurance Program (CAP) is a certification methodology developed by the Underwriters Laboratories company (UL) that adheres to their proprietary UL 2900 standards. Unfortunately, being a private and for-profit organization drastically reduces the community's trust in its certificates since they are not provable by third-party experts and do not adhere to the standardization and harmonization challenges.

\subsubsection{Certification de Sécurité de Premier Niveau}
The National Cybersecurity Agency of France developed certification de Sécurité de Premier Niveau (CSPN) in 2008 to verify the products' compliance with their security specifications in a timely manner. Unfortunately, this certification framework is standardized in France but not officially recognized by any other country, making it useless for international products.

%%%%%%%%%%%%%%%%%%%%%%%%%%%%%%%%%%%%%%%%%%%%%%%%%%%%%%%%%%%%%%%%%%%%%%%%%%%%%%%%%%

    


\begin{table}[htb]
    \centering
    \begin{tabular}{ |p{3cm}|p{2.3cm}|p{2cm}|p{2cm}|p{2cm}|p{2cm}|  }
         \hline
         \textbf{Challenges}&\textbf{CC}&\textbf{CPA}&\textbf{CAP}&\textbf{CSPN}&\textbf{ICSA Labs}\\
         \hline
         \textbf{Certification time (months)}* & 4-9 EAL2, 6-12 EAL3, 9-24 EAL4.
         & 6-18 & 6+ & 1-2 & 1-2\\
         \hline
         \textbf{Documentation} & Higher the EAL more complex the documentation & Hard to compare & Hard to compare & Hard to compare & Hard to compare\\
         \hline
         \textbf{Changes management} & Certificate valid only for specific patch and configuration & Any change is manually reviewed and can invalidate the certificate & Any change is manually reviewed and can invalidate the certificate & security patches invalidate the certificate & Any change is manually reviewed and can invalidate the certificate\\
         \hline
         \textbf{Context consideration} & \cmark & \xmark & Partially & \cmark & \cmark \\
         \hline
         \textbf{Monitoring} & \cmark & \cmark & \cmark & \xmark & \cmark \\
         \hline
         \textbf{Validity period} & Variable & Variable & 1 Year & 8 Weeks & 1 Year \\
         \hline
         \textbf{Labels completeness} & Only EAL info & \xmark & \xmark & \xmark & \xmark \\
         \hline
         \textbf{Mutual Recognition Agreements (MRA)} & Accepted by numerous countries & UK only & Not standard & France only & Not standard \\
         \hline
         \textbf{Cost (\$)}* & 75-200k EAL2, 110-250k EAL3, 150-350k EAL4 & 1300 per work day & Not standard & 25-30k & Not standard \\
         \hline
         \hline
         \multicolumn{6}{| c |}{\em*GAO analysis of data provided by laboratories.}\\
         \hline
    \end{tabular}
    \caption{Commercial Purpose Schemes Comparison}
    \label{Tab:comparison}
\end{table}


%%%%%%%%%%%%%%%%%%%%%%%%%%%%%%%%%%%%%%%%%%%%%%%%%%%%%%%%%%%%%%%%%%%%%%%%%%%%%%%%%%






\section{Configuration Changes}
A configuration change refers to modifying a component subject to change control. For example, some configuration changes that might interest an IoT system are the relocation of an asset (e.g. device, edge node, cloud server) or the change of some security property's attribute values.
Fields and attributes designated for change control are referred to as configuration information. There exist three main types of configuration change control \cite{IBMConfChange}:
\begin{description}
    \item[Configuration Change Tracking] Changes are tracked in the component record and automatically applied to the target attributes. This type of control can be paired with the other ones.
    \item[Revision Tracking] Based on revision versions that are incremented every time an update occurs. The revision records are used to support the configuration change tracking, which has to be present. This approach can also be combined with the approval-based change control, but when it is not, it is possible to change the configuration element directly in the component record.
    \item[Approval-Based Change Control] Configuration information cannot be altered without engineering change approval when this control change approach is involved. The evaluation involves using change or revision records to plan, approve and execute configuration changes; the used record is considered the engineering change record during this phase and provides core information about the history of past changes. Given the need for a configuration record, this approach can be combined with any other one.
\end{description}

Configuration changes represent a critical point in the cybersecurity certification frameworks because a certificate cannot predict how a system will be affected after one. Moreover, even though most developers are focusing on creating software that should not need a system restart after an update, the results are weak, and the system downtime represents another critical aspect for IoT systems. It should also be noted that an IoT system update includes the transferral of updated data and programs to all the devices composing it, resulting in a rather long downtime, depending on the system's topology and resources. Furthermore, downtime represents an issue for certification frameworks, especially for this thesis' proposed solution, due to the inability to start an evaluation process on the system.
To address the downtime matter, M. Ciok et al. recently proposed a platform to help developers with the system's architecture \cite{mateusz2022flex}.




\section{Looking Forward}
Certification schemes went a long way and passed through many iterations of improvements, optimizations and agreements; now, it is clear that no approach is perfect, especially when dealing with new generation systems that often include IoT layers in their stack. In addition, the IoT paradigm is much more dynamic than the cloud, both in the software and hardware, resulting in remarkably higher costs in terms of time and money; for this reason, the classic approaches seem not to be feasible anymore and result in being too restrictive for the IoT system manufacturers. The ideal certification scheme should be less rigorous in terms of assumptions and remove most costs (the manual phases). Such a scheme should deal with more variables, like unknown mechanisms, properties and attributes; a quick, ephemeral certification scheme that would verify only the strictly needed properties with the minimum effort to allow the system to keep operating right after remote configuration or context changes. Such an ideal scheme should address the following challenges:
\begin{enumerate}
    \item How should the properties be modelled now?\\ 
    As discussed in this chapter, current schemes rely on properties that are not expected to change, and their changes can lead to expensive, slow and redundant recertifications; finding new models for the system's properties is essential to allow manufacturers to keep updated their systems.
    
    \item Which of the other certification building blocks should change, and how?\\
    If the properties' model change, it is crucial to identify which of the other building blocks would be affected and which could be optimized.
    
    \item Remembering that some mechanisms might not be known, given a system and a property, what steps should be included in the new certification process to certify such property of the system?\\
    Non-functional mechanisms are usually known at certification time; this allows the laboratories to execute well-aimed tests to demonstrate the properties of the system exploiting such mechanisms. However, when dealing with dynamic systems, some configuration change could add or disable a mechanism which could not be easily detected; hence, it is important to design a scheme capable of dealing with this issue.
    
    \item How should the process deal with the limited resources of an IoT device/system in the testing phase?\\
    Downtime is another IoT system's major issue when dealing with recertifications. Some systems cannot afford to be disabled for long periods, but IoT devices have very limited resources. Minimizing the load and resorting to new testing methods could be needed to overcome this issue.
\end{enumerate}

The next chapter will focus on making the first step toward a new recertification scheme to address such challenges, specifically, challenges 1,2 and 4. It will introduce the scoring system and the trigger component that will make the process as fast and automated as possible; moreover, properties, attributes, evidence collection and certification will be remodelled and deeply discussed. Finally, the new models are introduced to obtain an extremely lightweight scheme that will allow manufacturers to recertify their systems with the minimum possible effort quickly.




\chapter{Methodology}
\label{cap3}
The recertification process is triggered when specific security conditions are met after a system's update; the components of the such process need to adapt to the new approach, and for the current step this thesis is taking, they will be modelled as follows:

\section{Attributes}
As for the State-of-the-art approaches, properties will still be modelled using attributes that define them at a lower level; more specifically, in the proposed approach, attributes define micro-properties (see properties paragraph). An attribute A is now modelled as a pair \((\hat{A}, [value_{min}, value_{max}])\) where \(\hat{A}\) is the abstract attribute and \(value_{i}\) is a possible value of A. It should be noted that the range of possible values is considered sorted by relevance, from the less relevant to the most relevant. For example, the value of the response time of a device is more relevant the lower it is; contrary, the value of a storage size is more relevant the bigger it is.

The score range of an attribute should be decided by an expert person having knowledge of the system and the context; such range should be balanced, setting the lowest score corresponding to the lowest relevant valid value and vice-versa.
The following linear mapping (mapping that maintains a constant ratio between points) operation allows mapping values between two ranges, in this case, values and scores:

\[q = \left \{ (p - A) \times \frac{D - C}{B - A} + C \bigm| [A, B] \in \mathbb{N} \wedge [C, D] \in \mathbb{N}  \right \} \]

Considering the range of valid attribute values from A to B, the range of scores from C to D, and the actual attribute value p, the mapping allows computation of the corresponding score value q.
A linear mapping requires two operations: a scaling operation to make the ranges the same size and an offset operation to align the ranges. Since scaling is always relative to the origin, the first step is to shift the range [A, B] to the origin by subtracting A from p; next, scale p by the ratio of the range sizes (D-C)/(B-A). Finally, shift this value to the start of the [C, D] range by adding C, resulting in the equation (ref). By rearranging the terms of the equation, it is possible to extract the scale and offset factors that compose it:

\[ scale = \frac{D - C }{B - A} \]
\[offset = -A \times \frac{D - C}{B - A} + C\]
\[q = \left ( p \times scale \right ) + offset\]

This equation maps the attribute's values range to the scores range spreading the values evenly (e.g., “equal importance” expressed with W[0] = 1/n, W[1] = 1/n, …, W[n]= 1/n). If the attribute's values are not numerical, they should be pre-emptively mapped to a list of indices; the indices list should then be used instead of the attribute's values. For example, for the micro-property Interface Control (ref), it is necessary to have a list of supported communication technologies that could include IEEE 802.11, Bluetooth and Ethernet connections; such a list might be mapped with an index list [1,2,3] where each index represents one of the elements.

The mapping also allows the system to compute the minimum values that each attribute of each micro-property needs to have to release a certificate with the minimum effort. Such values should then be gathered with a simple getter function.

\[ComputeMinValues([a_1, \dots , a_n], threshold)\]
\[GetMinValue(aj) \rightarrow {value_a}_j\]

\section{Properties}
The old properties are now referred to as micro-properties, and they have been organized into three categories referred to as macro-properties. Micro-properties are now modelled as a pair \(\left (\hat{p}, \left \{  (a_1, value), \dots , (a_n, value)\right \} \right )\) 
where \(\hat{p}\) is the abstract micro-property and \(\left ( a_i ,value \right )\) is a pair representing an attribute of the property and its value.

Macro-properties are intended to regroup micro-properties that influence similar aspects of the system; more specifically, every micro-property contributes to one of the macro-properties, namely, Confidentiality, Integrity and Availability. Macro-properties have a specific function in the proposed approach, helping with the micro-properties score attribution; since the thresholds will be based on the macro-properties, every micro-property's score will influence its macro-property final weighted score.

A macro-property of the system is then formalized as a triple \\ \( \left ( \hat{P}, \left \{(p_1, score), \dots , (p_n, score )\right \}, wscore \right ) \)
where \(\hat{P}\) is the abstract macro-property, \( \left (p_i, score \right ) \) is a pair representing a micro-property and its score, and wscore is the total weighted score of the macro-property. The weight determining the final result should depend on the context. 




\section{Trigger}
The trigger is a component detecting security changes in the system, evaluating them and acting in case these changes should impact crucial security properties. The trigger should also have continuous access to the system's information and certificates to detect changes accurately. The framework should access such changes through a \textit{Changes} function: 

\[Changes() \rightarrow \left \{ \left ( p_j, \left \{ a_k \right \} \right ) \right \} \]

Where \(p_j\) is a micro-property of the system and \( \left \{a_k \right \} \) is the set of specific attributes impacted by the update.

Such a function includes multiple background operations that the system has to deal with; these functionalities should probably include a monitoring framework for change detection. The monitoring should be invoked automatically every time a system (or context) update occurs. It should then compare the updated changes with the security properties and their attributes; any change over a security property should be immediately added to the Changes function output list. Moreover, the trigger component should also be notified about the availability of new changes.


\section{Evidence Collection Model}

Evidence collection is the process that, given a system and a property, gathers evidence proving the system holds such property through a series of tests. Such tests are now modelled as a tuple \( \left ( \hat{p}, \left (a_j, value_j, c_j, \left \{ cnd_k \right \} \right ) \right ) \)  where \( \hat{p} \) is the micro-property to test, \(a_j\) is an attribute of the micro-property, 
\(value_j\) is the expected value of the attribute, cj is the code to execute, and \( \left \{ cndk \right \} \) is the set of conditions that need to be satisfied by the test execution.

The evidence collection model is finally defined as a set of elements, one for each micro-property, modelled as \(  \left \{ \left \{ t_j \right \} {_{\hat{p}_i}} \right \} {_{\hat{P}_j}} \); where \( \hat{P}_j \) is a macro-property,  \( \hat{p}_i \) is a micro-property, \(a_j\) is an attribute of the micro-property, and \(t_j\) is the test to evaluate such an attribute.


\section{Certification Model Definition}
Once the trigger component starts the process by providing the system changes information, an important component that needs to be used for the recertification is the set of old certificates \(\mathcal{C}\); they contain all the demonstrated properties with their score and collected evidence prior to the update. Therefore, the proposed framework is expected to provide a function to obtain the certification model for the updated system; such function is formalized as follows:



Once the new certification model is computed, it will be possible to call the Execute function that will execute the tests on the system based on the given certification model returning the new certificate if all the requirements are satisfied.


The resulting certification model is defined as 
\(\mathcal{M} = \left ( \left ( P_1, P_2, P_3 \right ), \mathcal{E} \right )\) where \(P_1\) is the Confidentiality macro-property, \(P_2\) is the Integrity macro-property and \(P_3\) is the Availability macro-property, and \( \mathcal{E} \) is the evidence collection model.
 
The framework should be able to call the \textit{Execute} function using the certification model to deliver a new ephemeral certificate.

\[Ephemeral(\mathcal{C}, system_{t+1}) \rightarrow {\mathcal{M}}_{t+1} \]

Where \( \mathcal{C} = {c_0, \dots , c_t} \) and \(c_i\) is the certificate released at time \(i\); each certificate is modelled as a set of pairs \( (p, \{e_j\}) \) where \(p\) is a micro-property with the set of evidences supporting it.

\[ Execute(\mathcal{M}_{t+1}) \rightarrow \mathcal{C} \cup \{c_{t+1}\} \]
The system will finally be attributed a new certificate that will complement all the previously obtained ones.


\section{Assumptions}

For the next chapters, it is essential to clarify all  the major assumptions made during the design phase of the proposed certification process. As described in chapter \ref{cap2}, the current efforts in the IoT world are directed toward highly distributed systems (Edge computing) that involve some central computing nodes (Cloud computing). Therefore, when referring to IoT systems in the next chapters, it will be implied that the system will be composed of a Cloud layer, an Edge layer and an IoT layer. We note that the ToC model defined in chapter \ref{cap2} is still valid and has not been considered in this thesis work for simplicity, although in future, it could be modelled differently to better suit the new frameworks. Furthermore, it should be noted that this thesis' focus is on maintaining the certification's validity during the system's life cycle, mainly after some configuration change. Hence, when dealing with examples in the future chapters, it will be assumed that every system has already been fully certified with a standard certification model (e.g. Common Criteria) at the start of its functions. The proposed re-certification process should be aware of a threshold indicating the number of changes a system has to go through before the process is triggered. Such a threshold should be preemptively tested and precisely studied, possibly in the manual certification process, as it depends entirely on the context. It will be assumed that this value is known and irrelevant in future examples. There is also a major assumption in current standards that needs to be removed to apply the proposed solution; such assumption is to know a priori all the properties, attributes and mechanisms of the system. Only a part needs to be known; the rest should be found or inferred in some non-intrusive and lightweight manner. The risk with such an assumption is to debilitate the chain of trust and reduce the quality of the certification; if an attribute is discovered to have an invalid value, the entire certificate might be invalidated..
A couple of assumptions need to be made regarding the proposed certification model. Specifically, the trigger component is assumed to have access to the system's information, such as the old certificates; it is also assumed that the trigger has the capabilities to compute the system's changes over properties and their attributes. Another important point is the assumption of having an expert person knowing how to precisely balance the score values over the attributes given the knowledge of the system and the environment where it is deployed.


\chapter{Experiments}
\label{cap4}
This chapter will illustrate an example of applying the proposed certification scheme to real software, showing how the manufacturer could benefit from such a solution.

\section{Certification Target}
The software considered for this demonstration is OpenSSL, a robust, commercial-grade, full-featured Open Source Toolkit for the Transport Layer Security (TLS) protocol formerly known as the Secure Sockets Layer (SSL) protocol. The protocol implementation is based on a full-strength general-purpose cryptographic library, which can also be used stand-alone.
OpenSSL is descended from the SSLeay library developed by Eric A. Young and Tim J. Hudson []. The OpenSSL crypto library (libcrypto) implements various cryptographic algorithms used in various Internet standards. The services provided by this library are used by the OpenSSL implementations of TLS and CMS, and they have also been used to implement many other third-party products and protocols.
The functionality includes symmetric encryption, public key cryptography, key agreement, certificate handling, cryptographic hash functions, cryptographic pseudo-random number generators, message authentication codes (MACs), key derivation functions (KDFs), and various utilities.

OpenSSL was considered for this demonstration for two main reasons:
\begin{description}
    \item[Open-Source] It is open-source software, which allows browsing through the code (mainly written in the C language) publicly available in the official GitHub repository (https://github.com/openssl/openssl) and inspecting all the previous versions of the software with the relative update changes.
    \item[Vulnerabilities] Researchers found multiple exposed vulnerabilities of the software during the various years of deployment; in this thesis work, only two of them will be considered for the sake of the example, the first identified by the Common Vulnerability Scoring System (CVSS) as [CVE-2014-0224] and the second as [CVE-2016-7798].
    
    \begin{itemize}
        \item {[CVE-2014-0224]} allowed man-in-the-middle attackers [] to use a zero-length master key in certain OpenSSL-to-OpenSSL communications and consequently hijack sessions or obtain sensitive information via a crafted TLS handshake, also known as "CCS Injection" vulnerability [].
        \item {[CVE-2016-7798]} was a vulnerability of the OpenSSL Ruby gem, which used the same initialization vector for multiple encryptions in a specific configuration; this allowed attackers to decipher encrypted messages easily [].

    \end{itemize}
\end{description}

OpenSSL is not a product certified by third-party certification authorities with schemes like Common Criteria, but it will be considered as if it was for this demonstration. With standard approaches, the presence of such vulnerabilities would trigger the certificate revocation, obliging the manufacturer to fix the vulnerability and re-certificate the software with a full certification process, costing a considerable amount of money and time; instead, the proposed scheme would allow performing a faster and cheaper re-certification.

\section{Process Walkthrough}
This section will illustrate a complete walkthrough of the proposed certification scheme over a system based on OpenSSL that has to go through the OpenSSL cryptolibrary update.

\subsection{Trigger Phase}
The bug that exposed the first vulnerability ([CVE-2014-0224]) consisted of a bad implementation of the DTLS handshake protocol [], which allowed bad actors to defeat any encryption algorithm used effortlessly; hence, the influenced property is the ability to manage cryptographic keys securely.

The bug allowed entities to accept the ChangeCipherSpec message at timings different from the intended ones; hence, the attribute of the property that will be considered is the timing of the CCS message, which must satisfy two conditions to be accepted: i) there are no fragments from the handshake uncompleted (e.g. certificate validation) and ii) the next message received is Finished.

The bug that exposed the second vulnerability ([CVE-2016-7798]) consisted of a bad implementation of the Ruby Openssl gem when using the AES algorithm in Galois/Counter Mode (GCM) [], which allowed each initialization vector to be used multiple times; this resulted in weak encryption of data. The issue was caused by the initialization vector being set before the key when it should have been set after it. The attributes considered in this process are the encryption algorithm (AES-GCM), the satisfaction of the requirement of having a different initialization vector every time and the correlated certificate signing algorithm, which is also affected by this issue.

It is important to note that both the vulnerabilities described above affect the system's confidentiality; hence, confidentiality will be the only macro-property considered.

After the system receives the update, the trigger component analyses the changes, comparing the old certificates with the updated properties and attributes and extracting the following properties and attributes:
\begin{itemize}
    \item Ability to execute cryptographic mechanisms of appropriate strength and performance
    \begin{itemize}
        \item AES-GCM conditions
        \item AES-GCM key length
    \end{itemize}
    \item Ability to perform authenticated encryption algorithms
    \begin{itemize}
        \item DTLS handshake conditions
    \end{itemize}
    \item Ability to verify digital signatures
    \begin{itemize}
        \item Signature algorithm and key length
    \end{itemize}

\end{itemize}

After listing the impacted properties and attributes and making them available through the Changes function, the trigger signals a manual operator that the process requires attention.

\subsection{Manual Revision Phase}
An expert manual operator accesses the properties and attributes through the Changes function and deals with three main tasks:
\begin{itemize}
    \item According to technological standards, the operator has to assign to each attribute a list of values; in this example, they are:
    \begin{itemize}
        \item {[1]} (satisfied or not) for the AES-GCM conditions
        \item {[128, 192, 256]} for the key length of the AES-GCM algorithm
        \item {[1]} (satisfied or not) for the DTLS handshake conditions
        \item {[sha-1WithRSAEncryption, sha-224WithRSAEncryption, sha-256WithRSAEncryption, sha-384WithRSAEncryption, sha-512WithRSAEncryption]} for the signature algorithm and key length
    \end{itemize}
    
    \item According to his assumed knowledge, he assigns a range of scores that directly map to the values of each attribute:
    \begin{itemize}
        \item {[20]} for the AES-GCM conditions
        \item {[5, 10, 20]s} for the key length of the AES-GCM algorithm
        \item {[20]} for the DTLS handshake conditions
        \item {[5, 10, 20, 50, 70]} for the signature algorithm and key length
    \end{itemize}
    
    \item According to his assumed knowledge, he assigns a score threshold to the impacted macro-properties; in this case, to the confidentiality macro-property, he assigns a threshold of 80 and a weight of 1 (the final weighted sum is ignored for this example).
\end{itemize}

\subsection{Score Computation and Testing}
At this point, the system must compute the minimum value needed by each attribute to reach the threshold with the minimum possible effort. In this case, since the threshold is 80, each attribute needs a score of 20 and will be tested for the corresponding value:
\begin{itemize}
    \item 1 for the AES-GCM conditions
    \item 256 for the key length of the AES-GCM algorithm
    \item 1 for the DTLS handshake conditions
    \item sha-256WithRSAEncryption for the signature algorithm and key length
\end{itemize}

If the system holds such requirements, then the certificate can be released.

All tests were executed on a virtual machine running Linux Ubuntu version 22.04.1.
The tests for CVE-2014-0224 were executed using a client-server setup; the server was tested with OpenSSL version 0.9.8y, known for being vulnerable to the CCS injection attack, and version 3.0.2, which implements the security patch to the vulnerability. Both server instances were initialized with the same commands:

First, it is required to generate a new key and a certificate on the server that will be used to establish a secure SSL/TLS connection with the following command:
"openssl req -x509 -newkey rsa:2048 -keyout key.pem -out cert.pem -days 365 -nodes"; the command req is used to create a self-signed test certificate with the argument x509, and the other arguments are: i) newkey rsa:2048 generates a new private RSA key of size 2048 bits, ii) keyout copies the generated key into a file named key.pem, iii) out copies the generated certificate into a file named cert.pem, iv) days specifies the validity duration of the certificate and v) nodes avoids encrypting the generated private key.
Next, the server can be started using the following command:
"openssl s\_server -key key.pem -cert cert.pem -accept 44330 -www"; the command s\_server implements a generic SSL/TLS server which accepts connections from remote clients speaking SSL/TLS, and the arguments are: i) key specifies the private key to use, ii) cert specifies the certificate file to use, iii) accept specifies the port to listen on for connections, and iv) www sets the server to send a status message back to the client when it connects.

On the client side, executing a script designed to check if the server is vulnerable to the CCS injection exploit is sufficient; such a script is publicly available on [nmap.org website] and uses the open-source tool Nmap []. The script works by sending a 'ChangeCipherSpec' message out of order and checking whether the server returns a 'UNEXPECTED\_MESSAGE' alert record or not. Since a non-patched server would simply accept this message, the CCS packet is sent twice in order to force an alert from the server. The server is vulnerable if the alert type is different from 'UNEXPECTED\_MESSAGE'; the script is shown in appendix B []. The output provided by the script is the following for the vulnerable server test:  

\begin{verbatim}
Starting Nmap 7.80 ( https://nmap.org ) at 2022-09-17 19:23 CEST
Nmap scan report for localhost (127.0.0.1)
Host is up (0.00012s latency).

PORT      STATE SERVICE
44330/tcp open  unknown
| ssl-ccs-injection: 
|   VULNERABLE:
|   SSL/TLS MITM vulnerability (CCS Injection)
|     State: VULNERABLE
|     Risk factor: High
|      OpenSSL before 0.9.8za, 1.0.0 before 1.0.0m, and 1.0.1 before 1.0.1h
|      does not properly restrict processing of ChangeCipherSpec messages,
|      which allows man-in-the-middle attackers to trigger use of a zero
|      length master key in certain OpenSSL-to-OpenSSL communications, and
|      consequently hijack sessions or obtain sensitive information, via
|      a crafted TLS handshake, aka the "CCS Injection" vulnerability.
|           
|     References:
|       http://www.cvedetails.com/cve/2014-0224
|       https://cve.mitre.org/cgi-bin/cvename.cgi?name=CVE-2014-0224
|_      http://www.openssl.org/news/secadv_20140605.txt

Nmap done: 1 IP address (1 host up) scanned in 1.18 seconds
\end{verbatim}

and the following for the patched server test:
\begin{verbatim}
Starting Nmap 7.80 ( https://nmap.org ) at 2022-09-17 19:26 CEST
Nmap scan report for localhost (127.0.0.1)
Host is up (0.000079s latency).

PORT      STATE SERVICE
44330/tcp open  unknown

Nmap done: 1 IP address (1 host up) scanned in 0.17 seconds
\end{verbatim}

The test results show that the server running OpenSSL version 3.0.2 is no longer vulnerable to the CCS injection attack.


The test for CVE-2016-7798 was executed by replicating the bug using OpenSSL's Ruby gem; moreover, the execution was done using Ruby version 2.0.0p648, known for being affected by the bug, and Ruby version 3.1.0p0, which implements the security patch to the vulnerability. The script to check if the vulnerability is present is the following: 
\begin{lstlisting}[language=Ruby]
require 'openssl'
require 'set'

cipherTextsSet = Set[]
iterations = 1000000

for i in 1..iterations

    # Initialize Cipher
    cipher = OpenSSL::Cipher.new('aes-256-gcm')
    cipher.encrypt

    # Set random IV
    cipher.random_iv
 
    # Set key
    key = "00000000000000000000000000000000"
    cipher.key = key

    # Encrypt message
    message = 'this is a test message'
    cipherText = cipher.update(message) + cipher.final

    
    # Add encrypted text to set
    cipherTextsSet.add(cipherText)
end
\end{lstlisting}

The steps of the script are simple and are repeated one million times: i) initialization of the cipher object, ii) set a random initialization vector into the cipher, iii) set a 256bit key into the cipher, and iv) encrypt the message; the key and the message are always the same, and the initialization vector is theoretically always different. Therefore, the bug is reproducible when executing the operations in this same order, having the initialization vector set before the key. The encrypted messages are then inserted into a Set object, which filters out all the identical strings and allows counting the unique values obtained.

The output resulting from the vulnerable version is the following:
\begin{verbatim}
==> 1000000 messages have been encrypted.
==> The encrypted texts are all identical.
\end{verbatim}

and the output resulting from the patched version is the following:
\begin{verbatim}
==> 1000000 messages have been encrypted.
==> Succesfully generated 1000000 different encrypted texts.
\end{verbatim}

The test proves that the vulerability is not present anymore in the patched version.





\chapter{Conclusions}
\label{cap5}
This thesis aimed to address on of the main problems affecting today's standard certification processes when applied to highly dynamic and pervasive systems. More specifically, the objective was to take a first step toward the development of a certification scheme that would drastically reduce the redundancy and the time and resources needed by a system to obtain a certificate after some configuration change. The real-world-inspired scenario in Chapter 4 shows what the process would look like and how the proposed scheme would drastically decrease the effort needed compared to a traditional certification process such as the Common Criteria.

We introduced the scoring system and the trigger component and seamlessly added them to the already existing set of components, showing how they could be remodelled to fit together perfectly. Finally, in our experimentation, the scheme successfully delivered a certificate that could complement the previously obtained ones, avoiding redundancies in the process. However, the approach taken in this thesis might raise questions regarding the needed data and taxonomies during the manual review phase that, even if reduced, still represents a core element of the procedure. We note that this problem is tightly related to how standard approaches (e.g. Common Criteria) dealt with similar features developing mechanisms like the Protection Profiles, which are little more than a taxonomy regularly updated with new systems' data.

Further research is needed to develop the aspects that remained out of the focus of this thesis, such as the trigger component implementation, the automatisms connecting each phase with the next one and the data organization for the manual review phase. Moreover, the proposed scheme is intended to be fully integrated with the existing certification schemes since its main purpose is not to substitute them but to complement them, and additional research should be done to allow for it.

The work we presented first addressed how the issues relative to highly dynamic systems could be solved, allowing for simpler re-certification practices. The level of assurance granted by this kind of approach could finally allow systems manufacturers and administrators to improve their services and not worry about releasing security patches or upgrading their systems, causing the complete invalidation of certificates obtained through complex procedures. 

\appendix
\chapter{Properties Taxonomy}
\label{taxonomy}
The NIST officially published several technical and non-technical capabilities that an IoT system’s manufacturer should consider; It should be noted that the following security requirements have been officially classified under one of the three parts of the CIA triad by the NIST.

Device identification, which is the capability to identify the IoT device for multiple purposes and in multiple ways to meet organizational requirements, is composed of the following requirements:

\begin{description}
    \item[Identifier Management Support] Ability for device identification. Elements that may be necessary:
        \begin{itemize}
            \item Ability to uniquely identify the IoT device logically.
            \item Ability to uniquely identify a remote IoT device.
            \item Ability for the device to support a unique device ID (e.g., to allow it to be linked to the person or process assigned to use the IoT device).
        \end{itemize}
    \item[Device Authentication Support] Ability to support local or interfaced device authentication. Elements that may be necessary:
        \begin{itemize}
            \item Ability for the IoT device to identify itself as an authorized entity to other devices.
            \item Ability to verify the identity of an IoT device.
        \end{itemize}
    \item[Physical Identifiers] Ability to add a unique physical identifier at an external or internal location on the device authorized entities can access.
\end{description}

On the other hand, the device configuration is defined as The capability to configure the IoT device through logical or physical interfaces to meet organizational requirements and includes the following requirements:
\begin{description}
    \item[Logical Access Privilege Configuration] Ability for only authorized entities to apply logical access privilege settings within the IoT device and configure logical access privilege as described in Logical Access to Interfaces.
    \item[Authentication and Authorization Configuration] Ability for only authorized entities to configure IoT device authentication policies and limitations as described in Logical Access to Interfaces.
    \item[Interface Configuration] Ability for only authorized entities to configure aspects related to the device’s interfaces as described in Logical Access to Interfaces.
    \item[Display Configuration] Ability to configure content to be displayed on a device.
    \item[Device Configuration Control] Ability to change configurations on the IoT device based on operational events as described in Device Security and Cybersecurity Event Awareness.
    \begin{itemize}
        \item Ability to change the device’s software configuration settings.
        \item Ability for authorized entities to restore the device to a secure configuration defined by an authorized entity.
        \item Configuration settings for use with the Device Configuration capability including, but not limited to:
        \begin{itemize}
            \item Ability for authorized entities to configure the cryptography use itself, such as choosing a key length.
            \item Ability to configure any remote update mechanisms to be either automatically or manually initiated for update downloads and installations.
            \item Ability to enable or disable notification when an update is available and specify who or what is to be notified.
        \end{itemize}
    \end{itemize}
\end{description}

Every IoT system revolves around data, and protecting it is the core focus of cybersecurity; the NIST defines data protection as the capability to protect IoT device data to meet organizational requirements. Therefore, it is important to satisfy as many of the following requirements as possible to achieve a high level of data protection.

\begin{description}
    \item[Cryptography Capabilities and Support] Ability for the IoT device to use cryptography for data protection. Elements that may be necessary:
        \begin{itemize}
            \item Ability to execute cryptographic mechanisms of appropriate strength and performance.
            \item Ability to obtain and validate certificates.
            \item Ability to verify digital signatures.
            \item Ability to run hashing algorithms.
            \item Ability to perform authenticated encryption algorithms.
            \item Ability to compute and compare hashes.
        \end{itemize}
    \item[Cryptographic Key Management] Ability to manage cryptographic keys securely:
        \begin{itemize}
            \item Ability to generate key pairs.
            \item Ability to store encryption keys securely.
            \item Ability to change keys securely.
        \end{itemize}
    \item[Secure Storage] Ability for the IoT device, or tools used through the IoT device interface, to enable secure device storage. Elements that may be necessary:
        \begin{itemize}
            \item Ability to support encryption of data at rest.
            \begin{itemize}
                \item Ability to cryptographically store passwords at rest, as well as device identity and other authentication data.
                \item Ability to support data encryption and signing to prevent data from being altered in device storage.
            \end{itemize}
            \item Ability to secure data in device storage
                \begin{itemize}
                    \item Ability to secure data stored locally on the device.
                    \item Ability to secure data stored in remote storage areas (e.g., cloud, server, etc.).
                    \item Ability to utilize separate storage partitions for system and user data.
                \end{itemize}
            \item Ability to “sanitize” or “purge” specific or all data in the device.
        \end{itemize}
        
    \item[Secure Transmission] Ability to secure data transmissions sent to and from the IoT device. Elements that may be necessary:
    \begin{itemize}
        \item Ability to configure the cryptographic algorithm to protect data in transit.
        \begin{itemize}
            \item Ability to support trusted data exchange with a specified minimum strength cryptography algorithm.
            \item Ability to support data encryption and signing to prevent data from being altered in transit.
        \end{itemize}
        \item Ability to utilize one or more capabilities to protect the data it transmits from unauthorized access and modification.
        \item Ability to use cryptographic means to validate the integrity of data transmitted.
    \end{itemize}
\end{description}

Another important aspect is the security of logical access to the system's interfaces, which is the ability to require authentication to or identify the IoT device and to establish authentication and identification configuration and display requirements.

\begin{description}
    \item[Authentication Support] Ability to support authentication methods.
    \begin{itemize}
        \item Ability for the IoT device to require authentication prior to connecting to the device.
        \item Ability for the IoT device to support and require appropriate authentication
        \begin{itemize}
            \item Ability for the IoT device to require authentication prior to connecting to the device.
            \item Ability for the IoT device to support a second, or more, authentication method(s) through an out of band path such as temporary passwords or other one-use logon credentials, third-party credential checks, biometrics, text messages, hard tokens and manufacturer proprietary methods
        \end{itemize}
        \item Ability for the IoT device to hide or mask authentication information during authentication process.
        \item Ability for the IoT device to support a second, or more, authentication method(s) through an out-of-band path such as: Temporary passwords or other one-use credentials; Third-party credential checks; Biometrics; Text messages; Hard Tokens; etc.
    \end{itemize}
    
    \item[Authentication Configuration] Ability to require, or not require, authentication to, and/or identification of, the IoT device, and to establish authentication and identification configuration and display requirements. Elements that may be necessary:
    \begin{itemize}
        \item Ability to set and change authentication configurations, policies and limitations settings for the IoT device
        \begin{itemize}
            \item Ability to set the time period for how long the device will remain locked after an established configurable limit of unsuccessful login attempts has been met.
            \item Ability to disable or lock access to the device after an established number of unsuccessful login attempts.
            \item Ability to display and/or report the previous date and time of the last successful login authentication.
            \item Ability to automatically disable accounts for the IoT device after an establish period of inactivity
            \item Ability to support automatic logout of inactive accounts after a configurable established time period.
            \item Ability to support automatic removal of temporary, emergency and other special use accounts after an established time period.
        \end{itemize}
        \item Ability to authenticate external users and systems
        \item Ability to revoke their access.
    \end{itemize}
    
    \item[System Use Notification Support] Ability to support system use notifications.
    \begin{itemize}
        \item Ability to display to IoT device users an organizationally-defined system use notification message or banner prior to successful IoT device authentication. (e.g., the message or banner would provide privacy and security notices consistent with applicable federal laws, Executive Orders, directives, policies, regulations, standards, and guidance).
        \item Ability to create an organizationally-defined system use notification message or banner to be displayed on the IoT device
        \begin{itemize}
            \item Ability to edit an existing IoT device display.
            \item Ability to establish the maximum size (in characters, bytes, etc.) of the available device display.
        \end{itemize}
        \item Ability to keep the notification message or banner on the device screen until the device user actively acknowledges and agrees to the usage conditions
    \end{itemize}
    
    \item[Authorization Support] Ability to restrict all unauthorized interactions.
    \begin{itemize}
        \item Ability to identify authorized users and processes.
        \item Ability to differentiate between authorized and unauthorized users (physical and remote).
    \end{itemize}
    
    \item[Authentication And Identity Management] Ability to establish access to the IoT device to perform organizationally-defined user actions without identification or authentication.
    
    
    \item[Role Support And Management] Ability to establish unique, privileged, organization-wide, and other types of IoT device user accounts. Elements that may be necessary:
    \begin{itemize}
        \item Ability to create unique IoT device user accounts.
        \item Ability to assign roles to IoT device user accounts.
        \item Ability to identify unique IoT device user accounts.
        \item Ability to support a hierarchy of logical access privileges for the IoT device based on roles (e.g., admin, emergency, user, local, temporary, etc.).
        \begin{itemize}
            \item Ability to establish user accounts to support role-based logical access privileges.
            \item Ability to administer user accounts to support role-based logical access privileges.
            \item Ability to use organizationally-defined roles to define each user account’s access and permitted device actions.
            \item Ability to support multiple levels of user/process account functionality and roles for the IoT device.
        \end{itemize}
        \item Ability to apply least privilege to user accounts (i.e., to ensure that the processes operate at privilege levels no higher than necessary to accomplish required functions).
        \begin{itemize}
            \item Ability to create additional processes, roles (e.g., admin, emergency, temporary, etc.) and accounts as necessary to achieve least privilege.
            \item Ability to apply least privilege settings within the device (i.e., to ensure that the processes operate at privilege levels no higher than necessary to accomplish required functions).
            \item Ability to limit access to privileged device settings that are used to establish and administer authorization requirements.
            \item Ability for authorized users to access privileged settings.
        \end{itemize}
        \item Ability to support organizationally-defined actions for the IoT device.
        \begin{itemize}
            \item Ability to create organizationally-defined accounts that support privileged roles with automated expiration conditions.
            \item Ability to establish organizationally-defined user actions for accessing the IoT device and/or device interface.
            \item Ability to enable automation and reporting of account management activities.
            \item Ability to assign access to IoT device audit controls to specific roles or organizationally-defined personnel.
            \item Ability to control access to IoT device audit data.
            \item Ability to identify the user, process or device requesting access to the audit/accountability information (i.e., to ensure only authorized users and/or devices have access).
            \item Ability to establish conditions for shared/group accounts on the IoT device.
            \item Ability to administer conditions for shared/group accounts on the IoT device.
            \item Ability to restrict the use of shared/group accounts on the IoT device according to organizationally-defined conditions.
        \end{itemize}
        \item Ability to implement dynamic access control approaches (e.g., service-oriented architectures) that rely on:
        \begin{itemize}
            \item run-time access control decisions facilitated by dynamic privilege management.
            \item organizationally-defined actions to access/use device.
        \end{itemize}
        \item Ability to allow information sharing capabilities based upon the type and/or role of user attempting to share the information.
        \item Ability to restrict access to IoT device software, hardware, and data based on user account roles, used with proper authentication of the identity of the user to determine type of authorization.
    \end{itemize}
    
    
    \item[Limitations on Device Usage] Ability to establish restrictions for how the device can be used. Elements that may be necessary:
    \begin{itemize}
        \item Ability to establish pre-defined restrictions for information searches within the device.
        \item Ability to establish limits on authorized concurrent device sessions for:
        \begin{itemize}
            \item User accounts
            \item Roles
            \item Groups
            \item Dates
            \item Times
            \item Locations
            \item Manufacturer established parameters
        \end{itemize}
    \end{itemize}
    
    
    \item[External Connections] Ability to support external connections. Elements that may be necessary:
    \begin{itemize}
        \item Ability to securely interact with authorized external, third-party systems.
        \item Ability to allow for the user/organization to establish the circumstances for when information sharing from the device and/or through the device interface will be allowed and prohibited.
        \item Ability to establish automated information sharing to approved identified parties/entities.
        \item Ability to identify when the external system meets the required security requirements for a connection.
        \item Ability to establish secure communications with internal systems when the device is operating on external networks.
    \end{itemize}
    
    \item[Interface Control] Ability to establish controls for the connections made to the IoT device. Elements that may be necessary:
    \begin{itemize}
        \item Ability to establish requirements for remote access to the IoT device and/or IoT device interface including:
        \begin{itemize}
            \item Usage restrictions
            \item Configuration requirements
            \item Connection requirements
            \item Manufacturer established requirement
        \end{itemize}
        \item Ability to restrict use of IoT device components (e.g., ports, functions, microphones, video).
        \item Ability to logically or physically disable any local and network interfaces that are not necessary for the core functionality of the device.
        \item Ability to restrict updating actions to authorized entities.
        \item Ability to restrict access to the cybersecurity state indicator to authorized entities.
        \item Ability to restrict use of IoT device services.
        \item Ability to enforce the established local and remote access requirements.
        \item Ability to prevent external access to the IoT device management interface.
        \item Ability to control the IoT device’s logical interface (e.g., locally or remotely).
        \item Ability to change IoT device logical interface(s).
        \item Ability to control device responses to device input.
        \item Ability to control output from the device.
        \item Ability to support wireless technologies needed by the organization (e.g., Microwave, Packet radio (UHF/VHF), Bluetooth, Manufacturer defined)
        \item Ability to support communications technologies (including but not limited to):
        \begin{itemize}
            \item IEEE 802.11
            \item Bluetooth
            \item Ethernet
            \item Manufacturer defined
        \end{itemize}
        \item Ability to establish and configure IoT device settings for wireless technologies including authentication protocols (e.g., EAP/TLS, PEAP).
    \end{itemize}
\end{description}
    
    Given the intrinsic dynamicity of IoT systems and devices, software updates are frequent, and support mechanisms may need additional security properties. These requirements refer to the ability to update IoT device software and support mechanisms for such updates.
    \begin{description}
        \item[Update Capabilities] Ability to update the IoT device software within the device and/or through the IoT device interface. Elements that may be necessary:
        \begin{itemize}
            \item Ability to update the software by authorized entities only using a secure and configurable mechanism.
            \item Ability to identify the current version of the organizational audit policies and procedures governing the software update.
            \item Ability to restrict software installations to only authorized individuals or processes.
            \item Ability to restrict software changes/uninstallations to only authorized individuals or processes.
            \item Ability to verify software updates come from valid sources using an effective method (e.g., digital signatures, checksums, certificate validation, etc).
        \end{itemize}
        
        \item[Update Application Support] Ability to update the device’s software through remote (e.g., network download) and/or local (e.g., removable media) means
        \begin{itemize}
            \item If software updates are delivered and applied automatically:
            \begin{itemize}
                \item Ability to verify and authenticate any update before installing it
                \item Ability to enable or disable updating
            \end{itemize}
        \end{itemize}
    \end{description}
    
    
    %% skipping cybersecurity awareness %%
    
    Last but not least important is the device's security, defined as the capability to secure the IoT device to meet organizational requirements.
    
    \begin{description}
        \item[Secure Execution] Ability to protect the execution of code on the device. Elements that may be necessary:
        \begin{itemize}
            \item Ability to enforce organizationally-defined execution policies.
            \begin{itemize}
                \item Ability to execute code in confined virtual environments.
                \item Ability to separate IoT device processes into separate execution domains.
            \end{itemize}
            \item Ability to separate the levels of IoT device user functionality.
            \item Ability to authorize various levels of IoT device functionality.
        \end{itemize}
        
        
        \item[Secure Communication] Ability to securely initiate and terminate communications with other devices. Elements that may be necessary:
        \begin{itemize}
            \item Ability to enforce traffic flow policies.
            \item Ability to utilize standardized protocols.
            \item Ability to establish network connections.
            \item Ability to terminate network connections (e.g., automatically based on organizationally-defined parameters).
            \item Ability to de-allocate TCP/IP address/port pairings.
            \item Ability to establish communications channels.
            \item Ability to secure the communications channels.
            \item Ability to interface with DNS/DNSSEC.
            \item Ability to store and process session identifiers.
            \item Ability to identify and track sessions with identifiers.
        \end{itemize}
        
        \item[Secure Resource Usage] Ability to securely utilize system resources and memory. Elements that may be necessary:
        \begin{itemize}
            \item Ability to support shared system resources.
            \begin{itemize}
                \item Ability to release resources back to the system.
                \item Ability to separate user and process resources use.
                \item Ability to no one will read this.
            \end{itemize}
            \item Ability to manage memory address space assigned to processes.
            \item Ability to enforce access to memory space through the kernel.
            \item Ability to prevent a process from accessing memory space of another process.
            \item Ability to enforce configured disk quotas.
            \item Ability to continue operation when associated networks are unavailable (e.g., a smart smoke detector must still go off when a fire occurs even if it is not attached to the associated network).
            \item Ability to provide sufficient resources to store and run the operating environment (e.g., operating systems, firmware, applications).
            \item Ability to utilize file compression technologies (e.g., to provide denial of service protection).
        \end{itemize}
        
        
        \item[Device Integrity] Ability to protect against unauthorized changes to hardware and software. Elements that may be necessary:
        \begin{itemize}
            \item Ability to perform security compliance checks on system components.
            \item Ability to detect unauthorized hardware and software components.
            \item Ability to take organizationally-defined actions when unauthorized hardware and software components are detected (e.g., disallow a flash drive to be connected even if a USB port is present).
            \item Ability to store the operating environment (e.g., firmware image, software, applications) in read-only media (e.g., Read Only Memory).
        \end{itemize}
        
        \item[Secure Device Operation] Ability to operate securely and safely. Elements that may be necessary:
        \begin{itemize}
            \item Ability to keep an accurate internal system time.
            \item Ability to define various operational states.
            \item Ability to support various modes of IoT device operation with more restrictive operational states.
            \begin{itemize}
                \item “travel mode” for transit.
                \item “safe mode” for operation when some or all network security is unavailable.
                \item Others as determined necessary based on the purpose and goals for the IoT device.
            \end{itemize}
            \item Ability to define differing failure types.
            \item Ability to fail in a secure state.
            \item Ability to disable operations and/or functionality in the event of security violations.
            \item Ability to restrict components/features of the IoT device (e.g., ports, functions, protocols, services, etc.) in accordance with organizationally-defined policies.
            \item Ability to sense the environment and securely (i.e., preserving confidentiality, integrity, and availability of the device and its data) interface with the environment, either directly or through the IoT system. Examples include:
            \begin{itemize}
                \item Emergency shutoff mechanism
                \item Emergency lighting mechanism
                \item Fire protection mechanism
                \item Temperature and humidity mechanism
                \item Water damage protection mechanism
                \item Manufacturer defined capability
            \end{itemize}
        \end{itemize}
        
        
    \end{description}
    
    
    Moreover, Nasiri et al. \cite{nasiri2019security} defined a series of security requirements that manufacturers should consider (and develop) when designing IoT systems. These requirements are a high-level summary of the above that could be useful while showcasing examples of systems’ design. Although the research refers to healthcare IoT systems, many properties are still relevant for any kind of IoT system.
    
    \begin{description}
        \item[Identification and authentication] Identification guarantees the identity of all the entities before permitting them to interact with the resources of the IoT system. Authentication is the process of confirming the identity of a person or device before using the system resources. Devices and application authentication can prove that the interacting system is not an adversary and that data shared in networks is legal.
        
        \item[authorization (access control)] After user identity verification, access rights or privileges to resources should be determined so that different users can only access the resources required based on their tasks.

        \item[privacy] Privacy means that secrets and personal data of users should not be disclosed without consent. Therefore, the IoT system should be in accordance with privacy policies allowing users to control their private data.

        \item[accountability] In a health IoT system, accountability should ensure that the organization or individuals are obliged to be answerable or responsible for their actions in case of theft or abnormal event.

        \item[non-repudiation] Non-repudiation ensures that someone cannot deny an action that has already been done. It enables the users to prove the occurrence or non-occurrence of an event.

        \item[auditing] Auditing is the ability of a system to track and monitor actions continuously. All user activities should be recorded in sequential order, such as system login time and data modification.
        
        \item[data freshness] Data freshness means that data should be recent, ensuring that no old messages are replayed.

        

    \end{description}
    

\chapter{Scripts Used For Testing}
\section{CCS Injection Nmap Script}
\label{ccs}
\lstinputlisting[basicstyle=\footnotesize]{Files/ssl-ccs-injection}

\section{CVE-2016-779 Ruby Scripts}
\label{IV}
\textbf{Server}
\label{iv_server}
\lstinputlisting[basicstyle=\footnotesize, language=Ruby]{Files/IV/server.rb}
\textbf{Client}
\label{iv_client}
\lstinputlisting[basicstyle=\footnotesize, language=Ruby]{Files/IV/client.rb}

\newpage
\section{Signature Algorithm Ruby Scripts}
\label{sig}
\textbf{SocketIO}
\label{sig_io}
\lstinputlisting[basicstyle=\footnotesize, language=Ruby]{Files/Sig/socketIO.rb}
\textbf{Server}
\label{sig_server}
\lstinputlisting[basicstyle=\footnotesize, language=Ruby]{Files/Sig/server.rb}
\textbf{Client}
\label{sig_client}
\lstinputlisting[basicstyle=\footnotesize, language=Ruby]{Files/Sig/client.rb}

\section{AES-256-GCM Algorithm Ruby Scripts}
\label{aes}
\textbf{Server}
\label{aes_server}
\lstinputlisting[basicstyle=\footnotesize, language=Ruby]{Files/Sig/server.rb}
\textbf{Client}
\label{aes_client}
\lstinputlisting[basicstyle=\footnotesize, language=Ruby]{Files/Sig/client.rb}

%
%			BIBLIOGRAFIA
%
\begin{thebibliography}{00}

\bibitem{mell2011nist}Mell, P., Grance, T. \& Others The NIST definition of cloud computing. (Computer Security Division, Information Technology Laboratory, National …,2011)

\bibitem{atlam2017integration}Atlam, H., Alenezi, A., Alharthi, A., Walters, R. \& Wills, G. Integration of cloud computing with internet of things: challenges and open issues. {\em 2017 IEEE International Conference On Internet Of Things (iThings) And IEEE Green Computing And Communications (GreenCom) And IEEE Cyber, Physical And Social Computing (CPSCom) And IEEE Smart Data (SmartData)}. pp. 670-675 (2017)

\bibitem{khan2019edge}Khan, W., Ahmed, E., Hakak, S., Yaqoob, I. \& Ahmed, A. Edge computing: A survey. {\em Future Generation Computer Systems}. \textbf{97} pp. 219-235 (2019)

\bibitem{gokhale2018introduction}Gokhale, P., Bhat, O. \& Bhat, S. Introduction to IOT. {\em International Advanced Research Journal In Science, Engineering And Technology}. \textbf{5}, 41-44 (2018)

\bibitem{iorga2018fog}Iorga, M., Feldman, L., Barton, R., Martin, M., Goren, N., Mahmoudi, C. \& Others Fog computing conceptual model.  (2018)

\bibitem{sahni2018data}Sahni, Y., Cao, J. \& Yang, L. Data-aware task allocation for achieving low latency in collaborative edge computing. {\em IEEE Internet Of Things Journal}. \textbf{6}, 3512-3524 (2018)

\bibitem{yi2015fog}Yi, S., Hao, Z., Qin, Z. \& Li, Q. Fog computing: Platform and applications. {\em 2015 Third IEEE Workshop On Hot Topics In Web Systems And Technologies (HotWeb)}. pp. 73-78 (2015)

\bibitem{atlam2018fog}Atlam, H., Walters, R. \& Wills, G. Fog computing and the internet of things: A review. {\em Big Data And Cognitive Computing}. \textbf{2}, 10 (2018)

\bibitem{heck2010software}Heck, P., Klabbers, M. \& Eekelen, M. A software product certification model. {\em Software Quality Journal}. \textbf{18}, 37-55 (2010)

\bibitem{anisetti2017semi}Anisetti, M., Ardagna, C., Damiani, E. \& Gaudenzi, F. A semi-automatic and trustworthy scheme for continuous cloud service certification. {\em IEEE Transactions On Services Computing}. \textbf{13}, 30-43 (2017)

\bibitem{aslam2020fonac}Aslam, M., Mohsin, B., Nasir, A. \& Raza, S. FoNAC-an automated fog node audit and certification scheme. {\em Computers \& Security}. \textbf{93} pp. 101759 (2020)


\bibitem{baldini2016security}Baldini, G., Skarmeta, A., Fourneret, E., Neisse, R., Legeard, B. \& Le Gall, F. Security certification and labelling in Internet of Things. {\em 2016 IEEE 3rd World Forum On Internet Of Things (WF-IoT)}. pp. 627-632 (2016)

\bibitem{ardagna2015security}Ardagna, C., Asal, R., Damiani, E. \& Vu, Q. From security to assurance in the cloud: A survey. {\em ACM Computing Surveys (CSUR)}. \textbf{48}, 1-50 (2015)

\bibitem{infrastructure2002common}Infrastructure, P. \& Profile, T. Common criteria for information technology security evaluation. {\em National Security Agency}. (2002)

\bibitem{anisetti2013test}Anisetti, M., Ardagna, C., Damiani, E. \& Saonara, F. A test-based security certification scheme for web services. {\em ACM Transactions On The Web (TWEB)}. \textbf{7}, 1-41 (2013)

\bibitem{CSATrustSTAR}Cloud Security Alliance, Security CSATrust \& Assurance Registry (STAR). (\url{https://cloudsecurityalliance.org/star/}), Accessed: 2022-08-29

\bibitem{IBMConfChange}Configuration Change Overview. (\url{https://www.ibm.com/docs/en/mfnp/7.6.0?topic=changes-configuration-change-overview}), Accessed: 2022-08-29

\bibitem{mateusz2022flex}Mateusz Ciok, K., Pascual Espada, J. \& González Crespo, R. Flex-request: Library to make remote changes in the communication of IoT devices. {\em Expert Systems}. pp. e12994 (2022)

\bibitem{kaluvuri2014quantitative}Kaluvuri, S., Bezzi, M. \& Roudier, Y. A quantitative analysis of common criteria certification practice. {\em International Conference On Trust, Privacy And Security In Digital Business}. pp. 132-143 (2014)

\bibitem{keblawi2006applying}Keblawi, F. \& Sullivan, D. Applying the common criteria in systems engineering. {\em IEEE Security \& Privacy}. \textbf{4}, 50-55 (2006)

\bibitem{ICSAFw}Labs, I. ICSA Labs IoT Security Testing Framework. (\url{https://www.icsalabs.com/technology-program/iot-testing}), Accessed: 2022-08-29

\bibitem{surveyIOT}Matheu, S., Hernández-Ramos, J., Skarmeta, A. \& Baldini, G. A Survey of Cybersecurity Certification for the Internet of Things. {\em ACM Comput. Surv.}. \textbf{53} (2020,12), https://doi.org/10.1145/3410160

\bibitem{orangeBook}Latham, D. Department of defense trusted computer system evaluation criteria. {\em Department Of Defense}. \textbf{198} (1986)

\bibitem{ITSEC}Jahl, C. The information technology security evaluation criteria. {\em Proceedings-13th International Conference On Software Engineering}. pp. 306-307 (1991)

\bibitem{orangeBookDeath}Lipner, S. The birth and death of the orange book. {\em IEEE Annals Of The History Of Computing}. \textbf{37}, 19-31 (2015)

\bibitem{CTCPEC}Bacic, E. The Canadian trusted computer product evaluation criteria. {\em [1990] Proceedings Of The Sixth Annual Computer Security Applications Conference}. pp. 188-196 (1990)

\bibitem{CCProducts}Common Criteria Certified Products. (https://www.commoncriteriaportal.org/products/), Accessed: 2022-08-29

\bibitem{ardagna2014management}Ardagna, C., Asal, R., Damiani, E. \& Vu, Q. On the management of cloud non-functional properties: The cloud transparency toolkit. {\em 2014 6th International Conference On New Technologies, Mobility And Security (NTMS)}. pp. 1-4 (2014)

\bibitem{goertzel2007software}Goertzel, K., Winograd, T., McKinley, H., Oh, L., Colon, M., McGibbon, T., Fedchak, E. \& Vienneau, R. Software security assurance: a state-of-art report (sar). (INFORMATION ASSURANCE TECHNOLOGY ANALYSIS CENTER (IATAC) HERNDON VA,2007)

\bibitem{article}Ardagna, C., Asal, R., Damiani, E. \& Vu, Q. From Security to Assurance in the Cloud. {\em ACM Computing Surveys}. \textbf{48} pp. 1-50 (2015,7)

\bibitem{mahmud2018fog}Mahmud, R., Kotagiri, R. \& Buyya, R. Fog computing: A taxonomy, survey and future directions. {\em Internet Of Everything}. pp. 103-130 (2018)

\bibitem{macneil2006comply}MacNeil, I. \& Li, X. “Comply or Explain”: market discipline and non-compliance with the Combined Code. {\em Corporate Governance: An International Review}. \textbf{14}, 486-496 (2006)

\bibitem{anisetti2022multi}Anisetti, M., Ardagna, C. \& Bena, N. Multi-Dimensional Certification of Modern Distributed Systems. {\em IEEE Transactions On Services Computing}. (2022)

\bibitem{chen2013verification}Chen, M., Tan, T., Sun, J., Liu, Y., Pang, J. \& Li, X. Verification of functional and non-functional requirements of web service composition. {\em International Conference On Formal Engineering Methods}. pp. 313-328 (2013)

\bibitem{pub2005minimum}Pub, F. Minimum security requirements for federal information and information systems.  (2005)

\bibitem{nasiri2019security}Nasiri, S., Sadoughi, F., Tadayon, M. \& Dehnad, A. Security requirements of internet of things-based healthcare system: a survey study. {\em Acta Informatica Medica}. \textbf{27}, 253 (2019)

\bibitem{daemen1999aes}Daemen, J. \& Rijmen, V. AES proposal: Rijndael. (Gaithersburg, MD, USA,1999)

\bibitem{fischer2017dod}Fischer, T. DoD 5220.22-m data wipe method [US DOD wipe standard]. (Retrieved from Lifewire: https://www. lifewire. com/dod-5220-22-m-2625856,2017)

\bibitem{OpenSSLMan}Young, E. \& Hudson, T. OpenSSL Manual. , (\url{https://www.openssl.org/docs/man3.0/man7/crypto.html}), Accessed: 2022-09-13


\end{thebibliography}
% 
\end{document}


 
